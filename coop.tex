\section{Multi-Document Summarization}
In den letzten Jahren erzielten unterschiedliche Verfahren große Fortschritte beim automatisierten Zusammenfassen von mehreren Dokumenten zu einem repräsentativen Dokument.
Im Gegensatz zum Zusammenfassen von einzelnen Dokumenten enthalten mehrere Dokumente über ein Thema viele redundante Informationen, die so gefiltert werden müssen, dass diese Information nur einmal im Ausgabedokument dargestellt wird.
Wiedersprüchliche Informationen sind ebenfalls eine Herausforderung bei der Zusammenfassung.
Das Ziel von Multi-Document Summarization ist das Repräsentieren der wichtigen Aspekte aller Dokumente in einem einzelnen Dokument, um einen guten Überblick zu schaffen.

Insbesondere im Bereich der Multi-Review Summarization existieren viele unterschiedliche Ansätze.
Zum einen existieren extraktive Ansätze, wie zum Beispiel LexRank, ein unüberwachter Algorithmus der repräsentative Sätze für eine Bewertung basierend auf Ihrer Zentralität in einem TF-IDF gewichteten Graphen selektiert.
Zum anderen existieren abstraktive Ansätze wie zum Beispiel MeanSum, CopyCat oder COOP, die generative Ansätze verwenden um neuartige Sätze in den Bewertungen zu erzeugen.

MeanSum ist ein Autoencoder Modell mit LSTM Encoder und Decoder und errechnet zur Zusammenfassung von Bewertungen den Durchschnitt von den einzelnen Latentvektoren der Bewertungen.

CopyCat basiert auf einem Variational Autoencoder Modell, welches GRU Encoder und Decoder verwendet. 
Für jede Gruppe von Bewertungen berechnet CopyCat einen Latentvektor $c$ der die gesamte Semantik der Gruppe beschreibt. 
Weiterhin wird jede einzelne Bewertung einer Gruppe mit einem einzelnen Latentvektor $z$ beschrieben.
Bei der Generation von Durchschnittsbewertungen ermöglicht es CopyCat dem Decoder, die einzelnen Latentvektoren für die Bewertungen $z_i$, den allgemeinen Gruppenvektor $c$ und die Bewertungen $r_i$ an sich zu betrachten.
Durch den Zugriff auf die anderen Bewertungen $r_i$ kann der Decoder spezifische Worte von diesen \glqq kopieren\grqq{}  und somit übernehmen.



\subsection{Convex Aggregation for Opinion Summarization}\raggedbottom


\pagebreak
