% \section{Ergebnispräsentation}

% \begin{frame}
%   \frametitle{Grafiken}
%   Grafiken werden wie gewohnt eingebunden (nur ohne \texttt{figure}-Umgebung):
%   \begin{center}
% 	    \includegraphics[width=0.5\textwidth]{example-image-a}
%   \end{center}
% \end{frame}

% \begin{frame}{Grafiken mit Beschriftung}
% Oder alternativ mit \texttt{figure}-Umgebung:
% \begin{figure}
% \includegraphics[width=0.4\textwidth]{example-image-b}
% \caption{Ein Beispielbild}
% \end{figure}
% \end{frame}

% \begin{frame}{Quellenangaben}
% Falls es relevant ist, kann man eine Literatur-Quelle~\footfullcite{Krauthoff2017a} auf den Folien angeben.

% Dabei ist es praktisch, wenn die komplette Zitationsangaben auf der Folie selbst steht, ansonsten muss das Publikum bis zum Ende warten, um die Quellennummer aufzulösen.
% \end{frame}

% \begin{frame}{Notizen}
%   Notizen können mit \texttt{note} hinzugefügt werden. Diese werden gerendert, wenn in \texttt{master.tex} die Zeile mit \texttt{show notes on second screen} aktiviert wird.
  
%   Zum Präsentieren mit Notizen eigenen sich spezielle Programme wie PDF Presenter Console (pdfpc).
  
%   \note{Notizen}
% \end{frame}
