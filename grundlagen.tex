\section{Grundlagen}\raggedbottom
In diesem Kapitel werden die Grundlagen zu den verwendeten Technologien erklärt. Zunächst wird die in dieser Masterarbeit untersuchte Aufgabe des abstrakten Zusammenfassens erläutert.
Anschließend werden die Grundlagen zu neuronalen Netzen und insbesondere zu Deep Learning erklärt. 

Im folgenden Kapitel \ref*{transformer} wird anschließend aus den Grundlagen die verwendete Transformer Architektur, BERT und GPT-2 erklärt.


\subsection{Textzusammenfassung}
Das automatisierte Zusammenfassen von Texten ist ein Teilgebiet des NLP, welches sich mit dem Zusammenfassen von langen Texten zu einem kongruenten, kürzeren Text unter Beibehaltung von wichtigen Informationen befasst. 
Durch die zunehmenden Datenmengen wird automatisierte Textzusammenfassung immer relevanter, um akkurate Zusammenfassungen und Überblicke zu geben.
Automatisierte Textzusammenfassung lässt sich zum Beispiel bei der Generierung von Kurzzusammenfassungen in Zeitungen oder zur Generierung von Überschriften einsetzen.

Grundsätzlich wird beim automatisierten Zusammenfassen von Texten zwischen den extraktiven und den abstraktiven Methoden unterschieden.


\subsubsection{Extraktive Textzusammenfassung}
Extraktive Textzusammenfassung ist das Identifizieren und anschließende Extrahieren von wichtigen Phrasen oder Sätzen im Ursprungstext.
Hierbei werden die entsprechend ausgewählten Phrasen in der Zusammenfassung genauso wie sie im Ursprungstext vorkommen übernommen.
Die zu extrahierenden Phrasen oder Sätze werden mithilfe einer Scoringfunktion gefunden und später aneinander gereiht. Hierzu gibt es unterschiedliche Methoden und Metriken.

\subsubsection{Abstraktive Textzusammenfassung}
Abstraktive Textzusammenfassung versucht durch Interpretation und Verständnis des Ursprungtexts eine kurze, kongruente Zusammenfassung zu reproduzieren. 
Die Zusammenfassung soll alle wichtigen Informationen enthalten und als zusammenhängenden flüssigen Text erzeugt werden. Ein kongruenter Text wird erzeugt, da das Sprachmodell frei Sätze produzieren kann und nicht an vorher vorgegebene Sätze oder Phrasen gebunden ist, wie bei der extraktiven Zusammenfassung.

Große Fortschritte im Bereich der abstraktiven Textzusammenfassung ergaben sich in den letzten Jahren durch Sequence-To-Sequence Modelle. Diese encodieren den Eingabetext in eine Übergangsrepresentation und generieren aus dieser durch Decodierung eine Ausgangsrepresentation.

\subsubsection{Multi-Document Textzusammenfassung}
Eine große Herausforderung ist das Zusammenfassen von mehreren Dokumenten zum selben Thema zu einem einzigen Dokument. 
Die entstehende Zusammenfassung soll Anwendern einen guten und schnellen Überblick über eine große Anzahl an Dokumenten bieten. 
Die unterschiedlichen Dokumente enthalten diverse Informationen die nicht immer deckungsgleich sind. 
Somit ergibt sich die Herausforderung unterschiedliche Perspektiven in den jeweiligen Dokumenten zusammenfassend zu representieren.
Da sich unterschiedliche Standpunkte schlecht mittels extraktiven Methoden darstellen lassen, bieten sich bei der Multi-Document Textzusammenfassung abstraktive Methoden an.

In dieser Masterarbeit werden unterschiedliche Bewertungen zu Produkten oder Restaurants zusammengefasst. Insbesondere Bewertungen unterscheiden sich stark in Ihren Standpunkten und können positiv, negativ oder neutral sein und auf sehr spezifische Eigenschaften der entsprechenden Produkte eingehen.
Es ist eine große Herausforderung aus einer Menge aus Produktbewertungen eine allgemeingültige zusammenfassende Bewertung zu produzieren, die klar strukturiert, kongruent und gut lesbar die entsprechenden Inhalte wiedergibt.

\subsection{Deep Learning}

\subsection{Word Embeddings}


\pagebreak
