\section{Datensätze}\raggedbottom
\label{dataset}
%Datensätze mit Gruppentheorie erklären
Zum Training und zur Evaluation von neuronalen Netzen sind große Datensätze erforderlich. 
Für die Aufgabenstellung, Durchschnittsrezensionen aus Textbewertungen zu erzeugen, werden Datensätze mit mehreren Bewertungen und Zusammenfassungen zu einem Produkt benötigt.

Es bietet sich an, Bewertungen von großen Webportalen wie Amazon oder Yelp zu verwenden. 
Diese Portale haben zu den unterschiedlichsten Produkten und Restaurants zahlreiche Bewertungen.

Es existieren bereits bestehende Amazon und Yelp Datensätze mit menschlich erstellten Gold-Standard Zusammenfassungen. 
Die Variational Autoencoder Modelle lassen sich somit mit den Review Daten mit dem Trainingsziel der Rekonstruktion trainieren, um einen aussagekräftigen Latentraum mit Bewertungen und deren spezifischen Eigenschaften zu erlernen.
Anschließend können die trainierten Modelle auf den respektiven Testdatensätzen evaluiert werden, da jeweils Gold-Standard Zusamenfassungen vorliegen, mit denen sich mittels in Abschnitt \ref{evalmetric} erklärten Metriken die generierten Rezensionen bewerten lassen.

Der Umfang der zum Training und Testen verwendeten Datensätze ist in Tabelle \ref{dataset_table} angegeben.

\begin{table}[!h]
    \centering
    \begin{tabular}{@{}lcccccc@{}}
        \toprule
                              &            & \textbf{Amazon} &      &           & \textbf{Yelp} &      \\
                              & Train      & Dev             & Test & Train     & Dev           & Test \\ \midrule
    \# Produkte / Restaurants & 244.652    &   28            &  32  & 75.988    &  100          &  100    \\
    \# Rezensionen            & 13.053.202 &    224          &   256& 4.658.968 &  800          &  800 \\ \bottomrule
    \end{tabular}
    \caption{Enthaltene Rezensionen der einzelnen Datensätze \citep{coop}}
    \label{dataset_table}
    \end{table}

\subsection{Amazon Datensatz}
Der Amazon Review Datensatz \citep{brazinskas2020-unsupervised} umfasst zahlreiche Rezensionen zu den unterschiedlichsten Produkten die auf dem Amazon Marktplatz gelistet sind.
Die Rezensionen wurden von Amazon Nutzern erstellt und geben eine unabhängige Meinung über das erworbene Produkt ab.
Bei Betrachtung der einzelnen Bewertungen fällt auf, dass die meisten Bewertungen von Produkten eher objektiv formuliert sind. 
Die Bewertungen gehen oft auf unterschiedliche Aspekte der Produkte ein und erläutern positive, sowie negative Erfahrungen die auf einzelne Produkteigenschaften zurückzuführen sind.

\setlength{\fboxsep}{1em}

\begin{figure}[!h]
    \centering
    \scriptsize
    \framebox{

        \parbox{\columnwidth-4\fboxsep}{\textbf{Produkt:} NuGo Protein Bar, Vanilla Yogurt \\ \\ \textbf{Review:} Apart from the obvious nutritional value, the taste was surprisingly better than I anticipated. Normally, if I find soy-based nutritional bars tend to have a chalky or tree-bark-like texture and taste. I dare say, the texture reminds me of a healthy-not-too-sweet 'Rice Krispy Treat'.}

    }
    \caption{Beispiel Bewertung zu einem Produkt des Amazon Datensatzes}
\end{figure}

Die für das Training relevanten Bewertungen wurden aus den Kategorien Kleidung, Schuhe, Schmuck, Elektronikartikeln, Gesundheit- und Pflegeprodukte, Einrichtung und Küchenartikeln gewählt.
Gefiltert wurden die Bewertungen indem Produkte mit einer maximalen Länge von 128 Tokens gewählt wurden und nur Produkte mit über 10 existierenden Bewertungen ausgewählt wurden. 
Die entsprechenden Bewertungen sind bereits in Trainings- und Validierungsdaten gesplittet.
Die Validierungsdaten des Amazon Datensatz sind manuell erstellte Zusammenfassungen, die für den Dev / Test Split im Verhältnis von 28 / 32 vorliegen.
Somit ergibt sich insgesamt ein Trainingsdatensatz mit 244.652 Produkten und den zugehörigen 13.053.202 Bewertungen.
% Der Datensatz enthält somit diverse Produkte, bei denen jeweils unterschiedliche Eigenschaften wichtig sind.

% Der Amazon Datensatz wurde gefiltert und es wurden nur Produkte mit mindestens 10 Bewertungen, die eine Länge zwischen 20 und 70 Wörter haben, ausgewählt.
% Des Weiteren enthält der Amazon Datensatz manuell erstellte Zusammenfassungen, die für den Dev / Test Split im Verhältnis von 28 / 32 vorliegen.
% Die meisten Bewertungen im Amazon Datensatz sind eher objektiv formuliert und beziehen sich auf die einzelnen Produktspezifischen Eigenschaften.


% Der Amazon Review Datensatz \citep{brazinskas2020-unsupervised} umfasst 4.807.338 Bewertungen zu 192.742 Produkten. 
% Die Bewertungen wurden aus den Kategorien Kleidung, Schuhe, Schmuck, Elektronikartikel, Gesundheit- und Pflegeprodukte, Einrichtung und Küchenartikeln gewählt.
% Gefiltert wurden die Bewertungen indem eine maximale Länge von 150 Tokens gewählt wurde und nur Produkte mit über 50 existierenden Bewertungen ausgewählt wurden. 
% Die entsprechenden Bewertungen sind bereits in Trainings- und Validierungsdaten gesplittet.
% Der Datensatz enthält somit diverse Produkte, bei denen jeweils unterschiedliche Eigenschaften wichtig sind.

% Der Amazon Datensatz wurde gefiltert und es wurden nur Produkte mit mindestens 10 Bewertungen, die eine Länge zwischen 20 und 70 Wörter haben, ausgewählt.
% Des Weiteren enthält der Amazon Datensatz manuell erstellte Zusammenfassungen, die für den Dev / Test Split im Verhältnis von 28 / 32 vorliegen.
% Die meisten Bewertungen im Amazon Datensatz sind eher objektiv formuliert und beziehen sich auf die einzelnen Produktspezifischen Eigenschaften.

\subsection{Yelp Datensatz}
Der Yelp Review Datensatz \citep{meansum} enthält Rezensionen zu unterschiedlichen Gastronomiebetrieben und Unternehmen der Bewertungsplattform Yelp. 
Die von den Yelpnutzern erstellten Bewertungen unterschieden sich von den Amazon Bewertungen durch mehr Subjektivität in den Bewertungen.
Die Bewertungen enhtalten wesentlich mehr persönliche Details und Erfahrungen der Nutzer.
Das Variational Autoencoder Modell hat hier die Aufgabe, wichtige Informationen zu extrahieren und unwichtiges Rauschen in den Daten herauszufiltern, um eine gute Durchschnittsrepräsentation zu bestimmen.
Insbesondere in den Yelp Reviews befinden sich meistens unnötige Zusatzinformationen, die nicht zur Bewertung des Restaurants an sich beitragen, wie zum Beispiel das Erwähnen von privat ausgetragenen Geburtstagsfeiern. 

\begin{figure}[!h]
    \centering
    \scriptsize
    \framebox{

        \parbox{\columnwidth-4\fboxsep}{\textbf{Restaurant Id:} 87YsVbCN\_kfzheY79Fzjkg \\ \\ \textbf{Review:} Great experience! Went cake tasting for our wedding and Jessica was a huge help!! She brought out the tastings for us after giving us time to look through their amazing book. She explained every cake and filling and had great recommendations for mixing the fillings. The champagne cake with cream cheese filling was amazing!! Chose our cake and design within an hour. Great service and knowledge. Thank you Jessica!}

    }
    \caption{Beispiel Bewertung zu einem Restaurant des Yelp Datensatzes}
\end{figure}

Der Yelp Review Datensatz wurde ebenfalls nach einer maximalen Länge von 128 Tokens mit mindestens 10 existierenden Bewertungen je Unternehmen gefiltert und besteht somit aus 13.053.202 Bewertungen zu 244.652 Unternehmen, die bereits in Trainings- und Validierungsdaten gesplittet sind. 
Zusätzlich enthält der Datensatz 200 menschlich durch Amazon Mechanical Turk (AMT) Arbeiter erstellte Zusammenfassungen, die zu einem Dev / Test Verhältnis von 100 / 100 gesplittet werden. 
\pagebreak
