\pdfoptionpdfminorversion=5

\RequirePackage[table]{xcolor}


\documentclass[10pt, aspectratio=169, t ,hyperref={pdfpagelabels=false}]{beamer}
\usepackage{pgfpages}
%\setbeameroption{show notes on second screen=right}

\mode<presentation> {
    \usetheme{HHUD}
    \setbeamercovered{invisible}
}
\usepackage[ngerman]{babel}
\usepackage[utf8]{inputenc}
\usepackage{times}
\usepackage[T1]{fontenc}
\usepackage{amsmath}
\usepackage{subfigure}

\usepackage{hyperref}
\usepackage{xmpmulti}
\usepackage{multicol}
\usepackage{icomma}
\usepackage{booktabs}
\usepackage[backend=bibtex,style=verbose]{biblatex}
\bibliography{lit}
% If you want to exclude the appendix from the frame counter you have to use the appendixnumberbeamer package. But be aware that the current version causes a problem with the frame counter.
\usepackage{appendixnumberbeamer}

\usepackage{fancybox} % fuer die Rahmen
\usepackage{shortvrb}
\usepackage{url}
\usepackage{calc}

\usepackage{tikz}
\usetikzlibrary{positioning,matrix}
\usepackage{scrextend}

\usepackage{graphicx}
\usepackage{newfloat}
\DeclareFloatingEnvironment[fileext=lop]{Rezension}

\scriptsize
%Define a reference depth. 
%You can choose either relative or absolute.
%--------------------------
\newlength{\DepthReference}
\settodepth{\DepthReference}{g}%relative to a depth of a letter.
\setlength{\DepthReference}{1pt}%absolute value.

%Define a reference Height. 
%You can choose either relative or absolute.
%--------------------------
\newlength{\HeightReference}
\settoheight{\HeightReference}{T}
\setlength{\HeightReference}{4pt}


%--------------------------
\newlength{\Width}%

\newcommand{\ccolorbox}[2][red]%
{%
    \settowidth{\Width}{#2}%
    \setlength{\fboxsep}{1pt}%
    \colorbox{#1}%
    {%      
        \raisebox{-\DepthReference}%
        {%
                \parbox[b][\HeightReference+\DepthReference][c]{\Width}{\centering#2}%
        }%
    }%
}

\normalsize



\definecolor{HighlightColor}{HTML}{dc2626}
\definecolor{BackgroundColor}{HTML}{bfdbfe}
\normalsize



%Vergleich Optimus vs BiMeanVAE an Texten
\definecolor{ColorGray}{gray}{0.9}

%% Die folgenden Zeilen können auskommentiert werden, um vor jedem Kapitel eine Gliederungsfolie einzufügen
% \AtBeginSection[] {
%   \begin{frame}<beamer>
%     \thispagestyle{empty}
%     \frametitle{Gliederung}
%     \vspace{-5mm}
%     \tableofcontents[currentsection]
%   \end{frame}
% }

\usebackgroundtemplate{\includegraphics[width=\paperwidth]{fig/background_cd_2020}}

\newcommand{\backgroundNormal}{\usebackgroundtemplate{
    \includegraphics[width=\paperwidth]{fig/background_cd_2020}}}
\newcommand{\backgroundTitle}{\usebackgroundtemplate{
    \includegraphics[width=\paperwidth]{fig/background_heine}}}
\newcommand{\backgroundEmpty}{\usebackgroundtemplate{
    \includegraphics[width=\paperwidth]{fig/background_empty}}}
    
\setlength{\leftmargini}{9pt}
\setbeamersize{text margin left=25pt,text margin right=25pt} 
\setbeamertemplate{itemize/enumerate subbody end}{\vspace{.5\baselineskip}}

% % % % % % % % % %  CHANGE TOPIC AND AUTHOR INFORMATION HERE % % % % % % % % %
\newcommand{\abschluss}{Master}                              % HIER UNZUTREFFENDES LÖSCHEN
\title{\abschluss{}arbeit:\\Optimierung von Multi-Review Summarization unter Verwendung von auf
Transformern basierten Variational Autoencoder}                      % HIER DEN TITEL DER ARBEIT EINTRAGEN
\author{Lionel Schockenhoff}                                                       % HIER DEN NAMEN UND VORNAMEN EINTRAGEN
\date{DATUM}                                                                % HIER DAS PRÄSENTATIONSDATUM EINTRAGEN
% % % % % % % % % % % % % % % % % % % % % % % % % % % % % % % % % % % % % % % %
\institute{Institut für Informatik\\Heinrich-Heine-Universität Düsseldorf}
\subject{Informatik}

%
% Hier beginnt das Dokument
%
\begin{document}

\backgroundTitle
  \begin{frame}
    \thispagestyle{empty}
    \begin{columns}
    \column{0.4\paperwidth}
    {
    \footnotesize
    \color{hhuBlau}
    \put(20,-200){\insertdate}
    
    }
    \column{0.6\paperwidth}
    
    \color{hhuBlau}
    \LARGE \inserttitle\\[\baselineskip]
    
    \large \insertauthor
    \end{columns}
  \end{frame}
  \backgroundNormal

  % \begin{frame}
  %   \thispagestyle{empty}
  %   \frametitle{Gliederung}
  %   Diese Gliederung ist optional und nicht unbedingt empfohlen (vor allem bei Präsentationen unter einer halben Stunde; löschen in master.tex – es ist auch jedem vollkommen klar, dass die grobe Struktur Einleitung, Methoden, Ergebnisse, Fazit ist; die Struktur erwähnt man kurz mündlich, solange man beim Titel-Slide ist)
  %   \vspace{-5mm}
  %   \tableofcontents
  % \end{frame}

  % % % % % % % % % % Ab hier werden die LaTeX-Dateien der einzelnen Abschnitte eingefügt % % % % % % % % % %

  \section{Einleitung}\raggedbottom
Das automatisierte Zusammenfassen von Dokumenten ist im Bereich des Natural Language Processing (NLP) eine große Herausforderung.
Natural Language Processing ist ein Unterbereich der künstlichen Intelligenz, der sich mit dem maschinellen Verarbeiten von natürlicher Sprache auseinandersetzt. 
Aufgrund von neuen Methoden und immer größeren, komplexeren Netzwerken lassen sich hervorragende Ergebnisse in diversen NLP Aufgabenbereichen erzielen. 
Insbesondere lassen sich große, kostspielig vortrainierte Modelle mit vielen Parametern durch Transfer Learning in diversen speziellen Aufgabenbereichen einsetzen. 

Generative Pre-trained Transformers 2 (GPT-2) erzielte unter Verwendung von Transformern großartige Ergebnisse bei der Textgenerierung, unter anderem auch bei der abstraktiven Zusammenfassung von Texten. 
Die abstraktive Textzusammenfassung bezeichnet das Zusammenfassen von Texten zu einem kurzen, präzisen Text. 

\subsection{Motivation}
Im Bereich des E-Commerce und von Online-Vergleichsportalen entstehen eine große Anzahl an Rezensionen zu unterschiedlichen Produkten, Dienstleistungen und Anbietern.
Es ist für den Anwender eine Herausforderung sich einen repräsentativen Überblick über die einzelnen Rezensionen zu verschaffen. 
Ein gängiges Bewertungssystem ist das Berechnen des arithmetischen Mittels über die einzelnen Scores aller Bewertungen. 
Ein zusammengefasster Bewertungsscore ist nicht ausreichend repräsentativ für viele Produkte und stellt nicht spezifische Funktionen der einzelnen Produkte dar.
Hier bietet es sich an, ein Produkt durch eine generierte Rezension zu repräsentieren, die die anderen Rezensionen inkludiert und auf die verschiedenen wichtigen Aspekte der Produkte eingeht.
So kann ein Anwender sich zeitsparend einen ersten groben Überblick über bestimmte Produkte verschaffen.

\subsection{Ziel und Aufbau der Arbeit}
Ziel dieser Masterarbeit ist das unbeaufsichtigte abstraktive Zusammenfassen von mehreren Produktrezensionen mittels Variational Autoencodern zu einer repräsentativen Rezension.
Die generierten Zusammenfassungen sollten möglichst viele Aspekte der ursprünglichen Rezension aufgreifen, konsistent in ihrem Inhalt sein und einen guten Gesamtüberblick über die Rezensionen zu einem Produkt oder Dienstleistungen (zum Beispiel Restaurants) geben.
Als Datengrundlage gelten hier der Amazon-Review- und Yelp-Review-Datensatz, die zu Produkten und Dienstleistungen mehrere unterschiedliche diverse Rezensionen liefern.

Zunächst werden in Kapitel \ref{grundlagen} die verwendeten Grundlagen zu Textzusammenfassung, Deep Learning, Word Embeddings und Sequence To Sequence Modellen erläutert.
Anschließend werden in Kapitel \ref{transformer} die Transformermodelle Bidirectional Encoder Representations from Transformers (BERT) und GPT-2 erklärt, wobei detailliert auf die besonderen Merkmale wie Transformer Architekturen und Attention Layer eingegangen wird.
Der Attention Layer Aufbau ist insbesondere wichtig für die spätere Injektion der adaptierten Latentvektoren.

In Kapitel \ref{vae} wird der Aufbau und die mathematischen Details von Variational Autoencodern vermittelt.
Hier werden die zwei unterschiedlichen Variational Autoencoder Optimus und \textsc{BiMeanVAE} hergeleitet, die im weiteren Verlauf die Zusammenfassungen generieren.
Optimus basiert auf den beiden bekannten Sprachmodellen BERT und GPT-2. 
\textsc{BiMeanVAE} besteht aus einem bidirektionalem Long Short Term Memory (LSTM)-Encoder und einem LSTM-Decoder.

Im Anschluss werden in Kapitel \ref{dataset} die beiden Datensätze Amazon und Yelp vorgestellt, wobei die unterschiedlichen Aspekte und Herausforderungen der Datensätze dargestellt werden.

Danach werden in Kapitel \ref{multisum} mehrere aktuelle Verfahren zur Multi-Review Summarization vorgestellt. 
Convex Aggregation for Opinion Summarization (COOP) ist eins dieser Verfahren, welches unterschiedliche Kombinationen von einzelnen Latentvektoren untersucht.

In Kapitel \ref{pplm_chap} wird das vorgestellte COOP Verfahren durch ein Attributmodell weiter verbessert. 
Hierzu wird bei den Variational Autoencodern der Generationsprozess in Bezug auf ein Attributmodell optimiert.
Des Weiteren wird in diesem Kapitel eine neue Rankingfunktion zur Auswahl der Rezensionen eingeführt und die Hyperparameter für die Evaluation bestimmt.

Abschließend werden die konstruierten Modelle in Kapitel \ref{evalmetric} auf den Datensätzen evaluiert. Ebenfalls werden einzelne generierte Rezensionen untereinander verglichen.
In der abschließenden Zusammenfassung in Kapitel \ref{summary} wird ein Fazit und ein Ausblick auf weitere Forschungsgebiete gegeben.  




%Im NLP (\textbf{N}atural \textbf{L}anguage \textbf{P}rocessing) Bereich gibt es viele unterschiedliche Ansätze zur Textzusammenfassung von einzelnen Dokumenten.
%Da des Öfteren unterschiedliche Dokumente mit diversen Inhalten zu identischen Themen existieren, stellt sich die Frage wie diese unterschiedlichen Dokumente zu einem umfassenden Dokument zusammengefasst werden können.

%Ziel dieser Masterarbeit ist es mittels Variational Auto Encodern unterschiedliche Bewertungen zu Produkten und Restaurants abstraktiv zu einer allgemeingültigen Bewertung zusammenzufassen.




%Das Ziel dieser Arbeit ist mehrere Dokumente abstraktiv zu einem Dokument zusammenzufassen. Hierzu wird der Variational Auto Encoder OPTIMUS, der auf BERT und GPT-2 basiert verwendet. 

% Zunächst werden in Kapitel 2 die verwendeten Grundlagen zu Summarization, neuronalen Netzen, Deep Learning und den zugehörigen Subthemen erläutert. Ebenfalls wird auf die Einbettung von Wörtern eingegangen. Zur späteren Generierung von Question Answering Datensätzen werden in diesem Kapitel Abfragen mittels der Abfragesprache SPARQL an Knowledge Bases erklärt.

% Anschließend wird in Kapitel 3 die bidirektionale Transformer Architektur (BERT) erklärt, wobei detailliert auf die besonderen Merkmale wie Transformer Architekturen und Attention Layer eingegangen wird. Des Weiteren zeichnet sich BERT durch seine besonderen Pre-Trainingsmethoden, welche die Bidirektionalität des Modells ermöglichen, aus. 

% Im Anschluss werden in Kapitel 4, unter Verwendung von Python, Datensätze mit unterschiedlichen Vorgehensweisen generiert. Zum einen wird durch Abfragen auf Knowledge Bases ein Evaluationsdatensatz generiert, zum anderen wird durch maschinelle Übersetzung eines englischen Datensatzes ins Deutsche ein Trainings- und Evaluationsdatensatz erstellt. Es wird Bezug auf die besonderen Problematiken und Herausforderungen während der Entwicklung genommen. 

% Danach werden in Kapitel 5 zwei unterschiedlich vortrainierte Modelle mit den selbsterstellten Datensätzen und Kombinationen von anderen Datensätzen nachtrainiert. Insbesondere wird hier auf den Fine-Tuningsvorgang und die Wahl der unterschiedlichen Kombinationen der Datensätze eingegangen. 

% In Kapitel 6 werden die unterschiedlich trainierten Modelle evaluiert und untereinander verglichen. 
% In der abschließenden Zusammenfassung in Kapitel 7 wird ein Fazit und ein Ausblick auf weitere Forschungsgebiete gegeben.

% \pagebreak
% \subsection{Terminologie}

% \textbf{Adaptive Moment Estimation} (Adam) ist ein Optimierer für neuronale Netze.

% \textbf{Fine-Tuning} ist ein Trainingsprozess, der ein bereits vortrainiertes neuronales Netz an eine spezielle Aufgabenstellung anpasst und auf diese trainiert. Die bereits im Pre-Training gefundenen Parameter werden weiter angepasst.

% \textbf{Global Vectors for Word Representation} (GloVe) ein Verfahren für das Zuweisen von Vektorrepräsentationen zu Worten.

% \textbf{JavaScript Object Notation} (JSON) ist ein in Textform vorliegendes Datenformat.

% \textbf{Natural Language Processing} (NLP) beschreibt das maschinelle Verarbeiten und Analysieren von natürlichen Sprachen unter Verwendung von unterschiedlichen Techniken und Verfahren. 

% \textbf{Pre-Training} beschreibt den initialen Trainingsprozess eines neuronalen Netzes. Im Bereich des NLP wird Pre-Training verwendet, um ein allgemeines Sprachmodell zu trainieren, welches später auf individuelle Aufgabenstellungen nachtrainiert wird.

% \textbf{Tokenisierung} ist die Segmentierung eines Textes in eine Folge von Tokens. Die einzelnen Tokens sind Zeichenketten bestehend aus einem oder mehreren Zeichen.

\pagebreak


  \section{Grundlagen}\raggedbottom
\label{grundlagen}
In diesem Kapitel werden die Grundlagen zu den verwendeten Technologien vorgestellt. 
Zunächst wird die in dieser Masterarbeit untersuchte Aufgabe des abstrakten Zusammenfassens von mehreren Dokumenten erläutert.
Anschließend werden die Grundlagen zu neuronalen Netzen und insbesondere zu Deep Learning zusammengefasst dargestellt. 
Hier wird ebenfalls die Struktur von Sequence-To-Sequence Modellen und LSTM-Zellen erläutert.

Im folgenden Kapitel \ref*{transformer} wird anschließend aus den Grundlagen die verwendete Transformer Architektur, BERT und GPT-2 konkretisiert.


\subsection{Textzusammenfassung}
Das automatisierte Zusammenfassen von Texten ist ein Teilgebiet des NLP, welches sich mit dem Zusammenfassen von langen Texten zu einem kongruenten, kürzeren Text unter Beibehaltung von wichtigen Informationen befasst \citep{DBLP:journals/corr/abs-1904-00688}. 
Durch die zunehmenden Datenmengen wird automatisierte Textzusammenfassung immer relevanter, um akkurat aggregierte Zusammenfassungen und Überblicke zu geben.
Automatisierte Textzusammenfassung lässt sich zum Beispiel bei der Generierung von Kurzzusammenfassungen zu Dokumenten verwenden.
Grundsätzlich wird beim automatisierten Zusammenfassen von Texten zwischen den extraktiven und den abstraktiven Methoden unterschieden.

\subsubsection{Extraktive Textzusammenfassung}
Extraktive Textzusammenfassung ist das Identifizieren und anschließende Extrahieren von wichtigen Phrasen oder Sätzen aus dem Ursprungstext \citep{DBLP:journals/corr/ChengL16a}.
Hierbei werden die entsprechend ausgewählten Phrasen in der Zusammenfassung, genauso wie sie im Ursprungstext vorkommen, übernommen.
Die zu extrahierenden Phrasen oder Sätze werden mithilfe einer Scoringfunktion gefunden und später aneinander gereiht. Hierzu gibt es unterschiedliche Methoden und Metriken.

\subsubsection{Abstraktive Textzusammenfassung}
Abstraktive Textzusammenfassung versucht durch Interpretation und Verständnis des Ursprungtexts eine kurze, kongruente Zusammenfassung zu produzieren \citep{DBLP:journals/corr/NallapatiXZ16}. 
Die Zusammenfassung soll alle wichtigen Informationen enthalten und als zusammenhängender flüssiger Text erzeugt werden. 
Ein kongruenter Text wird dadurch erzeugt, dass das Sprachmodell frei Sätze produzieren kann und nicht an vorher vorgegebene Sätze oder Phrasen gebunden ist, wie bei der extraktiven Zusammenfassung.

Große Fortschritte im Bereich der abstraktiven Textzusammenfassung ergaben sich in den letzten Jahren durch Sequence-To-Sequence Modelle \citep{DBLP:journals/corr/NallapatiXZ16}. Diese encodieren den Eingabetext in eine Übergangsrepräsentation und generieren aus dieser durch Decodierung eine Ausgangsrepräsentation.

\subsubsection{Multi-Document Textzusammenfassung}
Eine große Herausforderung ist das Zusammenfassen von mehreren Dokumenten über dasselbe Thema zu einem einzigen Dokument. 
Die entstehende Zusammenfassung soll Anwendern einen schnellen, umfassenden Überblick über eine große Anzahl an Dokumenten bieten \citep{DBLP:journals/corr/abs-2011-04843}. 
Die unterschiedlichen Dokumente enthalten diverse Informationen, die nicht immer im Ergebnis und Vokabular deckungsgleich sind. 
Somit ergibt sich die Herausforderung, unterschiedliche Perspektiven in den jeweiligen Dokumenten zusammenfassend zu repräsentieren.
Da sich unterschiedliche Standpunkte nur unzureichend mittels extraktiven Methoden darstellen lassen, bieten sich bei der Multi-Document Textzusammenfassung die Verwendung abstraktiver Methoden an.
Damit kann eine kongruente Zusammenfassung generiert werden, die die unterschiedlichen Aspekte der Dokumente darstellt.

In dieser Masterarbeit werden unterschiedliche Rezensionen zu Produkten und Dienstleistungen zusammengefasst. 
Insbesondere Rezensionen unterscheiden sich stark in ihren Nutzerperspektiven und können positiv, negativ oder neutral sein und auf sehr spezifische Eigenschaften der entsprechenden Produkte eingehen.
Es ist eine große Herausforderung aus einer Menge aus Produktrezensionen eine repräsentative zusammenfassende Rezension zu produzieren, die klar strukturiert, kongruent, verständlich und zielgruppenbezogen die entsprechenden Inhalte wiedergibt.

\subsection{Deep Learning}
Deep Learning ist ein Teilbereich des Machine Learning, bei dem nach dem Vorbild für das menschliche Gehirn neuronale Netze verwendet werden. 
Neuronale Netze werden unter anderem beim Natural Language Processing eingesetzt. 
Die neuronalen Netze bestehen aus mehreren Layern, die sequentiell den Output des vorherigen Layers weiterverarbeiten. 
Das Ziel von Deep Learning ist, durch Training der neuronalen Netze, Repräsentationen beziehungsweise Approximationen für Funktionen in den Daten zu finden.
Zum Trainieren dieser Netze wird oft Backpropagation, ein Verfahren zur Berechnung der Gradienten in neuronalen Netzen und entsprechender Anpassung der Gewichte verwendet.

\subsection{Word Embeddings}
Word Embeddings werden im Natural Language Processing verwendet, um Wörter mit aussagekräftigen Vektoren zu repräsentieren. 
Hochdimensionale Objekte wie Wörter lassen sich mittels Word Embeddings in einen niedrigdimensionalen Raum darstellen und behalten dabei ihre semantischen Relationen bei. 
Auf diesen Vektoren lassen sich unterschiedliche arithmetische Operationen ausführen.

Bekannte kontextunabhängige Word Embedding Verfahren sind zum Beispiel word2vec \citep{word2vec} und GloVe (Global Vectors for Word Representation) \citep{glove}. 
Diese Verfahren berechnen für jedes Wort einen eindeutigen Vektor, der alle unterschiedlichen Eigenschaften dieses Wortes enthält.
Die errechneten Vektoren sind kontextunabhängig und für ein Wort wird stets der gleiche Vektor verwendet. 
Word2vec erlernt die entsprechenden Vektorrepräsentationen mittels eines SkipGram neuronalen Netzes \citep{word2vec}, GloVe hingegen über die nicht Nulleinträge einer Co-occurrence Matrix von Wörtern untereinander \citep{glove}. 

Kontextabhängige Verfahren, wie zum Beispiel ELMO \citep{elmo} oder BERT \citep{DBLP:journals/corr/abs-1810-04805}, generieren kontextabhängige Vektoren für Wörter und berücksichtigen dabei das Umfeld, in dem das einzelne Wort auftritt.
Bei diesen Verfahren werden zur Bestimmung eines Word Embeddings der gesamte Satz benötigt, um die erlernten kontextspezifischen Eigenarten für die Einbettung zu berücksichtigen. 

Verfahren wie zum Beispiel BERT nutzen besondere Tokenisierungsmethoden, die es erlauben, einzelne Wörter durch mehrere Tokens zu repräsentieren.
Dieses Splittingverfahren von Wörtern in Subtokens ist als WordPiece Verfahren \citep{wordpiece} für BERT und als BytePairEncoding \citep{bytepairencoding} für GPT-2 bekannt. 
Durch das Splitten in Subtokens und Erlernen von Einbettungen für diese lässt sich ein kleineres Wörterbuch erstellen. 
Da sich die Wörter stets in Subtokens zerlegen lassen, können Out-of-Vocabulary-Fehler vermieden werden.

\subsection{Sequence-To-Sequence Modelle}
Sequence-To-Sequence Modelle sind eine Gruppe von Deep Learning Modellen zur Sprachverarbeitung, die eine Textsequenz $(x_1,x_2,\ldots,x_n)$ bestehend aus den Tokens $x_i$ in eine andere Textsequenz $(y_1,y_2,\ldots,y_n)$ überführen \citep{DBLP:journals/corr/SutskeverVL14}.
Die Besonderheit bei diesem Modell liegt darin, dass die Länge der Eingabesequenz und der Ausgabesequenz unterschiedlich sein kann.
Die am meisten verwendeten Sequence-To-Sequence Modelle bestehen aus einer Encoder/Decoder Architektur.
Der Encoder und Decoder besteht jeweils aus sequentiellen Rekurrenten Neuronalen Netzen (RNN), die jeweils als Eingabe ein Token und den Hidden-State des vorherigen RNNs erhalten.
Somit summieren sich innerhalb des Encoders die Hidden-States auf und werden abschließend durch einen Hidden-State Vektor repräsentiert, der die Informationen aus der Eingabe enthält.
Der abschließende Hidden-State Vektor lässt sich wie folgt iterativ bestimmen, wobei $f$ für die entsprechende RNN-Zelle und $W$ für die Gewichtsmatrizen steht:
\begin{equation}
    h_t = f(W^{hh}h_{t-1}+W^{hx}x_t)
\end{equation}

Oft werden als RNN-Zellen Long Short Term Memory (LSTM)- oder Gated Recurrent Unit (GRU)-Zellen verwendet.
Der Decoder ist ebenfalls ein RNN, welcher aus dem Hidden-State Vektor des Encoders eine Ausgabesequenz generiert.

Um die Performance von Sequence-To-Sequence Modellen weiter zu optimieren, werden in Kapitel \ref{attention} Attention Modelle eingeführt.

\pagebreak
\subsection{\textbf{L}ong \textbf{S}hort \textbf{T}erm \textbf{M}emory}
\textbf{L}ong \textbf{S}hort \textbf{T}erm \textbf{M}emory (LSTM) Netze sind eine Untergruppe der Rekurrenten Neuronalen Netze (RNN), die es ermöglichen, Langzeitbeziehungen in Daten zu erlernen \citep{lstm}.
Rekurrente Neuronale Netze zeichnen sich durch ihre rekurrenten Verbindungen innerhalb desselben Layers aus, wodurch die nächste RNN-Zelle die vorherigen Informationen weiterverarbeiten kann.
Ein häufiges Problem ist das Modellieren von Langzeitbeziehungen mittels RNNs. 
LSTMs sind speziell darauf ausgelegt diese Langzeitbeziehungen abzubilden, indem sie stets einen Zellzustand mit übergeben.
Eine LSTM-Zelle hat nun die Möglichkeit diesen Zellstatus zu manipulieren, indem Informationen hinzugefügt oder gelöscht werden können.
Diese Manipulationen des Zellstatus werden durch drei Gates ermöglicht:
\begin{enumerate}
    \item Forget Gate - ermöglicht der Zelle alte Werte zu vergessen.
    \item Input Gate - steuert, zu welcher Gewichtung neue Eingabewerte in dem Hidden-State gespeichert werden.
    \item Output Gate - steuert, welcher Teil des Hidden-States in die folgende Zelle propagiert wird.
\end{enumerate}
% Erstens dem Forget Gate, welches der Zelle ermöglicht alte Werte zu vergessen.
% Zweitens dem Input Gate, welches steuert zu welcher Gewichtung neue Eingabewerte in dem Hidden-State gespeichert werden.
% Drittens dem Output Gate, welches steuert welcher Teil des Hidden-States in die folgende Zelle propagiert wird.

Durch die Anpassung des Hidden-States in den einzelnen Zellen können relevante Informationen anhand des Hidden-States durch die Zellen geleitet werden, wodurch lange Beziehungen in den Textsequenzen modelliert werden können \citep{lstmexplained}.
Das Propagieren durch eine Reihe von LSTM-Zellen sowie der Einfluss der einzelnen Gates ist in Abbildung \ref{lstm_chain} dargestellt.


\begin{figure}[h]
    
    \centering
    \includegraphics[width=\textwidth]{bilder/LSTM3-chain}
    \caption{Reihe von LSTM-Zellen mit entsprechenden Gates \citep{lstmexplained}}
    \label{lstm_chain}
\end{figure}

\pagebreak

  

\begin{frame}{Transformer }
  \begin{itemize}
      \item Encoder-Decoder Deep Learning Architektur für NLP Anwendungen
      \item Basieren auf Attention-Layern
      \item Parallelisierbarkeit zur Performancesteigerung
      \item Encoder wandelt eine Eingabesequenz $(x_1,\ldots,x_n)$ in eine kontinuierliche Übergangsrepräsentation $z=(z_1, \ldots, z_n)$ um
      \item Decoder kann aus Übergangsrepräsentation $z$ eine Ausgabesequenz $(y_1, \ldots, y_n)$ generieren
      \item Decoder ist autoregressiv
  \end{itemize}
\end{frame}

\begin{frame}{Transformer Encoder / Decoder}
  \begin{figure}
    \includegraphics[width=0.25\textwidth]{bilder/Transformer-Encoder-Decoder.png}
    \caption{Transformer Encoder (links) und Transformer Decoder (rechts)}
    \end{figure}
\end{frame}


\begin{frame}{Attention Layer}
  \begin{itemize}
    \item bestimmt Relevanz von Values zu Keys und Queries, um so den Fokus auf bestimmte Bereiche anzugeben
    \item $Attention(Q,K,V) = Softmax(\frac{Q\times K^T}{\sqrt{d_k}})\times V$
    \item Falls gesamte Eingabe aus einer Datenquelle $\Rightarrow$ Self-Attention-Layer: \begin{itemize} \item $Q = X \times W^{Q}, K = X \times W^{K}, V = X \times W^{V}$\end{itemize}
    \item Attention Funktionen lassen sich parallel ausführen $\Rightarrow$ Multi-Head Attention Layer  \begin{equation*}
      \begin{split}
      MultiHeadAttention(Q,K,V) &= [head_1, \ldots, head_h] \times W^{O} \\
      \text{mit } head_i &= Attention(QW_i^Q,KW_i^K,VW_i^V)
      \end{split}
  \end{equation*}
  \end{itemize}
 
\end{frame}

\begin{frame}{Bidirectional Encoder Representations from Transformers}
  \begin{itemize}
    \item bidirektionale kontextuelle Einbettung von Wörtern
    \item sequentielle Transformerencoderlayer (12 Basis-Modell)
    \item Pretrainings-Aufgaben: \begin{itemize} \item Masked Language Modeling \item Next Sentence Prediction \end{itemize}
    \item hervorragende Ergebnisse auf vielen NLP Benchmarks
    
  \end{itemize}
 
\end{frame}

\begin{frame}{Generative Pre-trained Transformer-2 (GPT-2)}

  \begin{itemize}
    \item autoregressives Sprachmodell auf Basis von Transformerdecodern
    \item 12 sequentielle Transformerdecoder (small Modell)
    \item Masked Self-Attention Layer
    \item Kontrollieren der Generierung nur möglich durch: \begin{itemize} \item Fine-Tuning \item Startsequenz \end{itemize}
    \item BytePairEncoding Verfahren

  \end{itemize}
 
\end{frame}

%MORE GRUNDLAGEN
  

\begin{frame}{Variational Autoencoder}
  \begin{itemize}
      
      \item Autoencoder: \begin{itemize}
        \item Encoder $\mathbf{E}:\mathbb{R}^n \rightarrow \mathbb{R}^m$ und Decoder $\mathbf{D}:\mathbb{R}^m \rightarrow \mathbb{R}^n$
        \item Ziel: Eingabedaten komprimieren und anschließend rekonstruieren
        \item Bottleneck - Dimension der Latentrepräsentation kleiner als die Eingabedimension $(m<n)$
        \item diskreter Latentraum
      \end{itemize}
      \item Variational Autoencoder: \begin{itemize}
        \item probabilistischer Ansatz
        \item Datensatz wird durch Wahrscheinlichkeitsfunktion $P(X)$ beschrieben
        \item Encoder speichert Daten als Verteilung im Latentspace $p_\theta (\mathbf{z\mid x})$
        \item Punkte können aus dieser Verteilung gezogen werden $z \sim p_\theta (\mathbf{z\mid x})$
        \item Decoder rekonstruiert Ausgabe aus Latentvektor $z$ mit $p_\theta (\mathbf{x\mid z})$
      \end{itemize}
  \end{itemize}
\end{frame}

\begin{frame}{Variational Autoencoder}
  \begin{figure}
    \includegraphics[width=0.7\textwidth]{bilder/vae.png}
    \caption{Variational Autoencoder}
    \end{figure}
\end{frame}

\begin{frame}{Evidence Lower Bound}
  \begin{equation*}
    p_\theta (\mathbf{z\mid x}) = \frac{p_\theta (\mathbf{x\mid z}) p_\theta (\mathbf{z})}{p_\theta(\mathbf{x})}
\end{equation*} nicht berechenbar, deshalb durch Inferenzmodell approximiert $q_\phi (\mathbf{z\mid x}) \approx p_\theta (\mathbf{z\mid x})$.

Optimierungsziel ist Minimierung des Rekonstruktionsfehlers des generativen Modells $\theta$ zwischen Eingabe und Ausgabedaten, sowie Minimierung der Kullback-Leibler-Divergenz  $D_{KL}(q_\phi(\mathbf{z\mid x})\parallel p_\theta(\mathbf{z\mid x}))$.


\end{frame}

\begin{frame}{Evidence Lower Bound}
  Für eine beliebige Wahl der Decoder Parameter $\phi$ gilt:
  \begin{align*}
    log(p_\theta(\mathbf{x})) &= \mathbb{E}_{ q_\phi(\mathbf{z\mid x}) } [log( p_\theta(\mathbf{x}) )] \nonumber \\
    &= \mathbb{E}_{q_\phi( \mathbf{z\mid x} )} \Bigl[ log \Bigl[ \tfrac{ p_\theta(\mathbf{x,z}) }{ p_\theta(\mathbf{z \mid x}) } \Bigr] \Bigr] \nonumber \\
    &= \mathbb{E}_{q_\phi( \mathbf{z\mid x} )} \Bigl[ log \Bigl[ \tfrac{ p_\theta(\mathbf{x,z}) }{ q_\phi( \mathbf{z\mid x} ) } \tfrac{ q_\phi( \mathbf{z\mid x} ) }{ p_\theta(\mathbf{z \mid x}) }\Bigr] \Bigr] \nonumber \\
    &= \underbrace{\mathbb{E}_{q_\phi( \mathbf{z\mid x} )} \Bigl[ log \Bigl[ \tfrac{ p_\theta(\mathbf{x,z}) }{ q_\phi( \mathbf{z\mid x} ) } \Bigr] \Bigr]}_{=\mathcal{L}_{\theta,\phi}(\mathbf{x}) \text{ (ELBO)}} + \underbrace{\mathbb{E}_{q_\phi( \mathbf{z\mid x} )} \Bigl[ log \Bigl[ \tfrac{ q_\phi( \mathbf{z\mid x} ) }{ p_\theta(\mathbf{z \mid x}) }\Bigr] \Bigr]}_{=D_{KL}(q_\phi(\mathbf{z\mid x})\parallel p_\theta(\mathbf{z\mid x}))} 
  \end{align*}
  mit $D_{KL}(q_\phi(\mathbf{z\mid x})\parallel p_\theta(\mathbf{z\mid x})) \geq 0$. Somit lässt sich die Lossfunktion $\mathbf{L}$ aufstellen.
  \begin{align*}
    \mathcal{L}_{\theta,\phi}(\mathbf{x}) &= log(p_\theta(\mathbf{x})) - D_{KL}(q_\phi(\mathbf{z\mid x})\parallel p_\theta(\mathbf{z\mid x})) \leq log(p_\theta(\mathbf{x})) \\
\mathbf{L}_{\theta,\phi} &= -\mathcal{L}_{\theta,\phi}(\mathbf{x}) = -log(p_\theta(\mathbf{x})) + D_{KL}(q_\phi(\mathbf{z\mid x})\parallel p_\theta(\mathbf{z\mid x}))  \\
    \hat{\theta},\hat{\phi} &= \underset{\theta, \phi}{\arg\min \ } \mathbf{L}_{\theta,\phi}
\end{align*}
\end{frame}

\begin{frame}{Reparametisierung}
  Samplen von $z \sim q_\phi(\mathbf{z\mid x})$ ist nicht deterministisch $\Rightarrow$ keine Backpropagation durchführbar,
  deshalb mittels Reparametisierung $z$ durch deterministische Funktion $f_\phi(x,\epsilon)$ darstellen mit $\epsilon$ als Hilfsvariable.
 
  \begin{align*}
      \mathbf{z} \sim q_\phi(\mathbf{z\mid x}) = \mathcal{N}(\mathbf{z;\mu,\sigma \mathcal{I}}) \nonumber \\
      \mathbf{z} = \mu + \sigma \times \mathbf{\epsilon} \text{ , mit } \mathbf{\epsilon} \sim \mathcal{N}(0,\mathcal{I}) 
  \end{align*}
  Für $q_\phi (\mathbf{z\mid x})$ wurde eine multivariate Gaussverteilung gewählt.

\end{frame}

\begin{frame}{Cyclical Annealing Schedule}
  KL-Vanishing Problem beim Trainieren von VAEs:\begin{itemize}\item VAE Modelle vernachlässigen globalen Kontext bei Generierung, da KL-Regularisierung sehr klein wird \item Somit sind erlernte Features nahezu identisch mit Normalverteilung und Decoder nutzt die Latentfeatures bei der Generierung nicht\end{itemize}
  $\Rightarrow$ $\beta$-Variational Autoencoder mit zyklischen Erhöhen des $\beta$ Wertes
  \begin{equation*}
    \mathbf{L}_{\beta} (\beta,\theta,\phi)= -\mathbb{E}_{z\sim q_\phi(\mathbf{z\mid x})}[log p_\theta (\mathbf{x\mid z})]- \beta D_{KL}(q_\phi(\mathbf{z\mid x}) \parallel p_\theta(\mathbf{z})) 
\end{equation*}
\end{frame}

\begin{frame}{Optimus}
  \begin{itemize}
    \item Deep Latent Variable Modell
    \item Ziel: Sätze in einem universellen Latentspace zu organisieren
  \end{itemize}
\begin{figure}[h]
    \centering
    \includegraphics[width=\textwidth]{bilder/optimus_scheme}
\end{figure}

% BERT und GPT-2 über eine VAE Architektur miteinander zu verbinden hat die Herausforderung, die unterschiedlichen Tokenisierungsschemen der einzelnen Modelle zu verwenden und den Latentvektor bei der Textgeneration von GPT-2 zu injizieren. 
% Die Eingabetokens von BERT verwenden das WordPiece Embedding Verfahren \citep{wordpiece} mit einer Vokabulargröße von 28.996 Tokens. 
% Die Ausgabe erfolgt über die Byte Pair Encoding Tokenisierung \citep{bytepairencoding} von GPT-2 mit einer Vokabulargröße von 50.260 Tokens. 
% Innerhalb des Netzwerkes wird im Latentvektor ein Token durch eine Einbettung $h_{Emb}$, die das Token, die Position und das Segment Embedding wiedergibt, repräsentiert.
% Um beim Training den Loss der Rekonstruktionsaufgabe zu berechnen, werden die Sätze mit beiden Tokenisierungen tokenisiert.


% Als Latentvektor $z \in \mathbb{R}^P$ wird die gepoolte Ausgabe des letzten Hiddenlayers $h_{[CLS]} \in \mathbb{R}^H$ von BERT mit einer Gewichtsmatrix multipliziert $W_{E} \in \mathbb{R}^{P\times H}$ gewählt. Somit kann ein Latentvektor wie folgt bestimmt werden $z = W_{E}h_{[CLS]}$.

\end{frame}

\begin{frame}{Optimus}
  \begin{itemize}
    \item BERT und GPT-2 haben unterschiedliche Tokenisierungen $\Rightarrow$ beide benutzen 
    \item Latentvektor $z \in \mathbb{R}^P$ ist gepoolte Ausgabe des letzten Hiddenlayers $h_{[CLS]} \in \mathbb{R}^H$ von BERT mit einer Gewichtsmatrix multipliziert $W_{E} \in \mathbb{R}^{P\times H}$ gewählt $\Rightarrow$ $z = W_{E}h_{[CLS]}$.
  \end{itemize}

  \begin{figure}[h]
    \centering
    \includegraphics[width=11cm]{bilder/latent_optimus}
\end{figure}

\end{frame}

\begin{frame}{\textsc{BiMeanVAE}}
 \begin{itemize}
\item Bidirektionaler LSTM Encoder (anschließend Mean Pooling Layer)
\item LSTM Decoder
\item Hiddensize 512
\item trainiert mit Cyclical Annealing Verfahren
 \end{itemize}
\end{frame}

%MORE GRUNDLAGEN
  

\begin{frame}{Datensätze}
  Es werden Datensätze mit mehreren Bewertungen und Zusammenfassungen zu einem Produkt benötigt.
  \begin{itemize}   
      \item Amazon-Datensatz: \begin{itemize}
        \item Objektivere Bewertungen
        \item Kleidung, Schuhe, Schmuck, Elektronikartikel, Gesundheitsprodukte, Einrichtung, Küchenartikel
        \item 3 Gold-Standard Zusammenfassungen
      \end{itemize}
      \item Yelp-Datensatz: \begin{itemize}
        \item Subjektivere Bewertungen
        \item 1 Gold-Standard Zusammenfassung
      \end{itemize}
  \end{itemize}

  \begin{table}[!h]
    \centering
    \scalebox{0.8}{
    \begin{tabular}{@{}lccc|ccc@{}}
        \toprule
                              &            & \textbf{Amazon} &      &           & \textbf{Yelp} &      \\
                              & Train      & Dev             & Test & Train     & Dev           & Test \\ \midrule
    \# Produkte / Dienstleister & 244.652    &   28            &  32  & 75.988    &  100          &  100    \\
    \# Rezensionen            & 13.053.202 &    224          &   256& 4.658.968 &  800          &  800 \\ \bottomrule
    \end{tabular}}
   % \caption{Enthaltene Rezensionen der einzelnen Datensätze}

    \end{table}
\end{frame}


%MORE GRUNDLAGEN
  

\begin{frame}{Multi-Review Summarization}
  Sei $R= \{x_i \}$ ein Datensatz von Bewertungen mit einzelnen Bewertungen $x=(x_1,...,x_{\| x \|})$, die aus einer Sequenz aus Tokens bestehen.
Ziel ist es für ein gegebenes Produkt $p$ und die entsprechenden Bewertungen $R_p \subseteq R$ eine Zusammenfassung $s_p$ zu generieren, die alle relevanten Informationen faktisch korrekt repräsentiert.
\begin{itemize}   
  \item Extraktive Modelle: \begin{itemize}
    \item LexRank (TF-IDF gewichteter Graph)
  \end{itemize}
  \item Abstraktive Modelle (VAE): \begin{itemize}
    \item MeanSum \begin{itemize}
      \item $z_p = \bar{z} \text{, mit } z=\{z_{p_1},z_{p_2},...z_{p_k}\}$
    \end{itemize}
    \item CopyCat \begin{itemize}
      \item GRU Encoder und Decoder
      \item Latentvektor $c$ für Semantik einer Reviewgruppe
    \end{itemize}
    \item Convex Aggregation for Opinion Summarization (COOP)
  \end{itemize}
\end{itemize}
\end{frame}

\begin{frame}{Convex Aggregation for Opinion Summarization}
  \begin{itemize}
    \item Untersucht unterschiedliche Kombinationen von einzelnen Latentvektoren
    \item Bewertungen, die aus den normalen Durchschnittslatentvektoren errechnet wurden ähneln sich oft
    \item Normaler Durchschnittslatentvektor hat geringe $L_2$-Norm $\| z \|$
  \end{itemize}
  \begin{figure}
    \includegraphics[width=0.85\textwidth]{bilder/coop.png}
    \caption{Latentraum Z mit den entsprechenden generierten Bewertungen X}
  \end{figure}
\end{frame}


\begin{frame}{Convex Aggregation for Opinion Summarization}
  COOP formuliert Optimierungsproblem, um die beste Kombination von einzelnen Latentvektoren zu finden:
\begin{alignat*}{2}
    \max_z              &\quad&  Overlap(R_p, D_\theta(z))    & \\
    \text{unter der Nebenbedingung: } &\quad&  z = \sum_{i=1}^{|R_p|} w_i z_i \\
                         &\quad&  \sum_{i=1}^{|R_p|}w_i=1,                        &\quad \forall w_i \in \mathbb{R}^+
\end{alignat*}

mit $Overlap(X,Y) = \text{ROUGE-1}_{F_1}(X,Y)$
\end{frame}


\begin{frame}{Convex Aggregation for Opinion Summarization}
  Suche der einzelnen Kombinationen wird auf die Potenzmenge der Eingabebewertungen $R_p$ beschränkt. 
\begin{align*}
z_p = \frac{1}{|R_p^{'}|} \sum_{i=1}^{R_p^{'}}z_i \text{, mit } R_p^{'} \in 2^{R_p} \setminus  \{ \emptyset\} 
\end{align*}

\begin{itemize}
\item ausdrucksstarke Latentvektoren
\item aktuell State-of-the-Art Ergebnisse
\item höherer Informationsgehalt als Standard Durchschnittslatentvektoren
% \item weitere Optimierungen in dieser Masterarbeit $\Rightarrow$
% \begin{itemize}
%   \item Latentvektoren adaptieren für höheres Input-Output-Overlap
%   \item neue Input-Output-Overlap Ranking Funktion
% \end{itemize}
\end{itemize}
\end{frame}



%MORE GRUNDLAGEN
  

\begin{frame}{Kontrollierbare Textgenerierung mit Sprachmodellen}
  COOP Suche der Kombinationen von Latentvektoren erzielt bereits gute Resultate.
  Es werden die VAE Modelle Optimus und \textsc{BiMeanVAE} verwendet, welche zur Generierung den kombinierten Latentvektor $z$ mittels Decoder $p_\theta(x|z)$ zu einem Text $\hat{x}$ rekonstruieren.

  \begin{block}{$\Rightarrow$ Fragestellung}
    Lassen sich durch weitere Optimierungen noch bessere Ergebnisse erzielen?
  \end{block}

  \begin{itemize}
    \item Optimierung des Generierungsprozess durch ein Attributmodell
    \item Verbesserte Rankingfunktion
  \end{itemize}
\end{frame}

\begin{frame}{Attributmodell}
  \begin{itemize}
    \item Unkontrollierte Sprachmodelle modellieren Texte über die Wahrscheinlichkeit $p(X)$ für eine Sequenz $X=(x_0,...,x_{\| x \|})$
    \item Kontrollierbare Generierung von Texten $p(x|a)$ \begin{itemize} \item $a$ ist kontrollierbares Attribut in Bezug auf den generierten Text $x$ \end{itemize}
    \item Satz von Bayes  $ \Rightarrow \  p(x|a)\propto p(a|x)p(x)$ 
    \item Attributmodell $p(a|x)$ bewertet einen Satz $x$ auf den Besitz eines Attributs $a$ mit einer Wahrscheinlichkeit
  \end{itemize}
\end{frame}

\begin{frame}{Attributmodell}
  Transformer Sprachmodelle verwenden bei der Generierung, Key-Value Paare der Attention-Layer, die die bereits generierten Teilsequenzen in einer Vergangenheitsmatrix $H_t = [(K_t^{(1)},V_t^{(1)}), \ldots , (K_t^{(n)},V_t^{(n)})]$ speichern.

Somit lässt sich bei der Generierung auf bereits vorher berechnete Key-, Value-Werte zurückgreifen, um somit effizienter Text zu generieren.

  \begin{itemize}
    \item Gradienten über Log-Likelihood des normalen Sprachmodells $log(p(x))$ und der Log-Likelihood des Attribut-Modells $log(p(a|x))$
    \item Veränderung der Vergangenheitsmatrix $\tilde{H}_t = (H_t+\Delta H_t)$, um Wahrscheinlichkeit zu erhöhen gewünschte Tokens zu erzeugen
    \item $\Delta H_t$ wird schrittweise durch den Gradienten des Attribut-Modells bestimmt
  \end{itemize}
  \begin{align*}
\Delta H_t \leftarrow \Delta H_t + \alpha \frac{\nabla_{\Delta H_t} \text{log }p(a|H_t+\Delta H_t)}{\| \nabla_{\Delta H_t} \text{log }p(a|H_t+\Delta H_t)\|^\gamma}
\end{align*}
\end{frame}

\begin{frame}{Attributmodell}

\begin{itemize}
\item Modifizierte Ausgangsverteilung $\tilde{p}_{t+1}$ wird mit der nicht modifizierten Ausgangsverteilung $p_{t+1}$ kombiniert
\end{itemize}
Samplen von den kombinierten Verteilungen durch:
\begin{align} \label{combine_pplm}\hat{x}_{t+1} \sim \frac{1}{\beta} (\tilde{p}_{t+1}^{\text{ }\gamma} \cdot p_{t+1}^{1-\gamma})\end{align}
\begin{itemize}
  \item $\gamma$ bestimmt den Einfluss des unmodifizierten Sprachmodells auf die Ausgabe
  \item $\gamma \rightarrow 1$ Ausgabe zur Verteilung des modifizierten Sprachmodells 
  \item $\gamma \rightarrow 0$  Ausgabe zur Verteilung des unmodifizierten Sprachmodells
  \end{itemize}
\end{frame}


\begin{frame}{Bag of Words Attribut-Modell}
  
  Sei $\{w_0, \ldots, w_n\}$ eine Gruppe von Tokens, die zuvor gewählt wurden und $p_{t+1}$ die Ausgabeverteilung über die Tokens des Sprachmodells.

  \textbf{Loss} über die Summe der Wahrscheinlichkeiten der einzelnen vorhergesagten Wörter.
  \begin{align*}
      \text{log }p(a|x) = \text{log }(\sum_{i=0}^n p_{t+1}[w_i])
  \end{align*}

  \begin{block}{Wie wird das Bag of Words Modell erzeugt?}
    \begin{itemize}
\item $k$ häufigsten Wörter der Eingaberezensionen
\item Softmax Gewichtung der Wörter nach Vorkommen
\item Keyword-Extraktionsmethoden wie zum Beispiel YAKE
    \end{itemize}
    
  \end{block}
  
\end{frame}

\begin{frame}{Bag of Words AttributModell}
\begin{table}[h!]
  \centering
  \begin{tabular}{@{}llll@{}}
  \toprule
                  Modell   & \multicolumn{3}{c}{Amazon}              \\ 
  \textbf{\textsc{BiMeanVAE} }    & \textbf{R1} & \textbf{R2} & \textbf{RL} \\ \midrule
%  \textsc{BiMeanVAE} &             &             &             \\
  Baseline        & 39.50       & 8.62    &  22.79     \\
  $\quad$ k = 50       &  40.72    &   \textbf{9.06}    &   \textbf{23.40}   \\
  $\quad$ k = 150  &  \textbf{40.75}   &    9.03  &  \textbf{23.40}   \\
  $\quad$ k = 500 &  \textbf{40.75}   &    9.03  &  \textbf{23.40}   \\
  $\quad$ k = 50 + Softmax    &  40.72    &   \textbf{9.06}    &    \textbf{23.40}   \\
  $\quad$ k = 150 + Softmax  &  \textbf{40.75}   &    9.03  &  \textbf{23.40}   \\
  $\quad$ k = 500 + Softmax   &  \textbf{40.75}   &    9.03  &  \textbf{23.40}   \\
  %$\quad$ PRUNING??   &  40.75   &    9.03  &  23.40  \\
  $\quad$ YAKE &  40.72   &     9.00  &  \textbf{23.40}   \\ \bottomrule
  \end{tabular}
  \caption{Ergebnisse für die unterschiedlichen Attributmodelle mit \textsc{BiMeanVAE} auf dem Amazon Dev-Datensatz. Die besten Ergebnisse sind \textbf{fett} markiert.}
\end{table}

\end{frame}

\begin{frame}{OPTIMUS mit Attributmodell}
  Optimus generiert bereits durch Injektion des Latentvektors in die Vergangenheitsmatrix $p(x|z)$.

  Da der Latentvektor $z$ in die Vergangenheitsmatrix $H_t$ injeziert wird, wird der Latentvektor direkt optimiert und $\Delta z$ ergibt sich durch die folgende Iteration:
\begin{align*}
    \Delta z \leftarrow \Delta z + \alpha \frac{\nabla_{\Delta z} \text{log }p(a|z+\Delta z)}{\| \nabla_{\Delta z} \text{log }p(a|z+\Delta z)\|^\gamma}
\end{align*}
  
\end{frame}

\begin{frame}{\textsc{BiMeanVAE} mit Attributmodell}
  \textsc{BiMeanVAE} ist ein Variational Autoencoder, bestehend aus einem bidirektionalem LSTM Encoder, gepaart mit einem LSTM Decoder.

Der LSTM Decoder erhält als Eingabe den Latentvektor $z$. Der Hiddenstate $h_t$ und Cellstate $c_t$ wird aus dem Embedding des Latentvektor $z$ initialisiert und anschließend autoregressiv über die LSTM Architektur aktualisiert.

Um diesen LSTM Decoder mit einem Attributmodell zu optimieren, bieten sich drei unterschiedliche Ansätze:

\begin{enumerate}
    \item Optimieren über den Latentvektor $z$
    \item Optimieren des vorherigen Hiddenstates $h_t$ vor der nächsten Berechnung
    \item Optimieren des vorherigen Cellstates $c_t$ vor der nächsten Berechnung
\end{enumerate}
  
\end{frame}


\begin{frame}{\textsc{BiMeanVAE} mit Attributmodell}
  \begin{table}[h!]
      \centering
      \begin{tabular}{@{}llll@{}}
      \toprule
                      Modell   & \multicolumn{3}{c}{Amazon}              \\ \midrule
      \textbf{\textsc{BiMeanVAE}}    & \textbf{R1} & \textbf{R2} & \textbf{RL} \\ \midrule
    %  \textsc{BiMeanVAE} &             &             &             \\
      Baseline        & 39.50       & \underline{8.62}     &  22.79     \\
      $\quad$ optimze $z$        &   \underline{40.04}     &   8.35    &    \textbf{23.64}   \\
      $\quad$ optimze $h_t$      &  39.96   &    8.34  &  22.94  \\
      $\quad$ optimze $c_t$      &  \textbf{40.81}   &     \textbf{8.91}  &   \underline{23.49}    \\ \bottomrule
      \end{tabular}

  \end{table}
 Die größte Leistungssteigerung erzielt eine Optimierung der Variable $c_t$.
  \begin{align}
      \Delta c_t \leftarrow \Delta c_t + \alpha \frac{\nabla_{\Delta c_t} \text{log }p(a|c_t+\Delta c_t,z)}{\| \nabla_{\Delta c_t} \text{log }p(a|c_t +\Delta c_t ,z )\|^\gamma} \label{opt_lstm}
  \end{align}
  
\end{frame}

\begin{frame}{Moverscore}
  \begin{itemize}
    \item Evaluationsmetrik, zum semantischen Vergleich von Inhalten zweier Textsequenzen
    \item nutzt kontextuelle Einbettungen zum Vergleich (DistilBERT)
    \item hohe Ähnlichkeit mit menschlicher Bewertung
    \item Metrik: Word-Mover-Distance
  \end{itemize}
\end{frame}

\begin{frame}{Moverscore Ranking}
  Neue Rankingfunktion, die ROUGE-1,ROUGE-2,ROUGE-L und den Moverscore miteinbezieht, um semantische Ähnlichkeit besser abbilden zu können.
  \begin{align*}
    \small
    \begin{split}
    Input\text{-}Output\text{-}Overlap(\hat{x_i}, R) &= x \cdot \text{R1}(\hat{x_i}, R)+y\cdot \text{R2}(\hat{x_i}, R)+z\cdot \text{RL}(\hat{x_i}, R) 
\end{split}\\
\small
\begin{split}
    \text{SCORE}(\hat{x_i}) &= Input\text{-}Output\text{-}Overlap(\hat{x_i}, R) + v \cdot Moverscore(\hat{x_i}, R) 
\end{split}
\end{align*}
\end{frame}

\begin{frame}{Hyperparameter Optimierung mittels Dev-Datensatz}
  
 
  \begin{table}[!h]
    \centering
    \scalebox{0.74}{
    \begin{tabular}{@{}lcccc|cccc@{}}
    \toprule
    DEV-Dataset                    & \multicolumn{4}{c}{Amazon} & \multicolumn{4}{c}{Yelp} \\ 
    \textbf{Method} & \textbf{R1} & \textbf{R2} & \textbf{RL} & \textbf{MV} & \textbf{R1} & \textbf{R2} & \textbf{RL} & \textbf{MV}\\ \midrule 
    
    \textit{COOP + Attribute Model}        &      \multicolumn{3}{l}{Stepsize $\alpha= 0.05$}             &        &   & &     \\
    $\quad$ Optimus          &  36.31 & 7.17& 20.75&\textbf{56.82} &  36.31   &   \underline{8.12}   & \underline{20.02}  &  56.71    \\ 
    %$\quad$ \textsc{BiMeanVae}   &  37.93 & 7.19& 21.85&56.77 &     &      &   &    \\[0.2cm] % TEST STEP 05
    $\quad$ \textsc{BiMeanVae} & 36.40 & \underline{7.19} & \underline{21.65} & \textbf{56.55} &36.16 &7.51 &20.49 & 56.91\\ [0.2cm] % DEV STEP 05
    
    \textit{COOP + Attribute Model}        &         \multicolumn{3}{l}{Stepsize $\alpha= 0.1$}            &        &   & &     \\
    $\quad$ Optimus            & \underline{36.40}  & \textbf{7.78}&\textbf{20.99} &56.78 &   36.31  &  8.08    &  19.91 & 56.67  \\ 
    %$\quad$ \textsc{BiMeanVae} &  \textbf{38.16} & 7.30 & \underline{22.00} & 56.78&     &      &   &     \\[0.2cm]  %TEST STEP 1
    $\quad$ \textsc{BiMeanVae}& 36.45 & 7.08 & 21.64 & \underline{56.54} & \underline{36.33}&\underline{7.61} & \textbf{20.50} &56.92 \\[0.2cm]  %DEV STEP 1


    \textit{COOP + Attribute Model}        &      \multicolumn{3}{l}{Stepsize $\alpha= 0.3$}                &        &        &   & &     \\
    $\quad$ Optimus            & \textbf{36.43}  & \underline{7.64}& \underline{20.76} &\underline{56.79} &   \underline{36.47}  &    \textbf{8.19}  & \textbf{20.21}  & \textbf{56.78}  \\ 
    %$\quad$ \textsc{BiMeanVae} & 37.98  & \textbf{7.56} & 21.95& \underline{56.92} &     &      &   &     \\[0.2cm] %TEST STEP 3
    $\quad$ \textsc{BiMeanVae} & \underline{36.55} & \textbf{7.32} & \textbf{21.78} & 56.43 & 36.33 & 7.53 & 20.39& \textbf{57.00}\\[0.2cm] %DEV STEP 3


    \textit{COOP + Attribute Model}        &    \multicolumn{3}{l}{Stepsize $\alpha= 0.5$}            &        &   & &     \\
    $\quad$ Optimus            & 35.84  &6.93 & 19.75& 56.23&   \textbf{36.54}  &   8.05   &19.84   & \underline{56.74}  \\ 
    %$\quad$ \textsc{BiMeanVae} &   \underline{37.99}&\underline{7.44} &\textbf{22.24} & \textbf{57.00}&     &      &   &     \\ %TEST STEP 5
    $\quad$ \textsc{BiMeanVae}& \textbf{36.58} & 7.07 & 21.39 & 56.49 & \textbf{36.45} & \textbf{7.67} & \underline{20.49}& \underline{56.94} \\ \midrule % DEV STEP 5

    \textit{COOP}              &         &         &        &        &        & &   &    \\
    % $\quad$ Optimus  $^{\star}$           & 35.98 & 7.17    & 20.16 & - & 35.51  & 7.84   & 19.27 & -\\ 
    % $\quad$ \textsc{BiMeanVae} $^{\star}$  & 36.30 &  6.81 &  21.11 & - & 36.16  & 7.25  & 20.09 & -\\ 
    $\quad$ Optimus        & 35.97 & 7.16 & 20.15 & 56.60 & 35.50  & 7.84  & 19.26 & 56.63\\  %CHECK OPTIMUS YELP
    $\quad$ \textsc{BiMeanVae} &  35.67 &6.53 & 21.07 & 56.27 & 36.16 & 7.25 & 20.09 & 56.78\\ 
    	\bottomrule
    
    \end{tabular}}
    
\end{table}
  
\end{frame}

\begin{frame}{Hyperparameter Optimierung mittels Dev-Datensatz}

  \begin{itemize}
    \item Leistungssteigerung in allen Bereich durch Verwendung des COOP + Attributmodells
    \item Aufgrund der unterschiedlichen Aspekte der Modelle und Datensätze, jeweils individuelle Hyperparameter
  \end{itemize}
  
  \begin{table}[h!]
    \centering
    \begin{tabular}{@{}lccccc|ccccc@{}}
    \toprule
    & \multicolumn{5}{c}{Amazon} & \multicolumn{5}{c}{Yelp} \\ 
    
             &$\alpha$ & R1(x)  & R2(y)  & RL(z) & MV(v)&$\alpha$ &  R1(x)  & R2(y)  & RL(z) & MV(v) \\ \midrule

    Optimus            & 0.3 & 1.0 & 0.5 & 1.5 & 3.5 & 0.5 & 3.5 & 1.0 & 0.0 & 2.5 \\
    \textsc{BiMeanVAE} & 0.3 & 3.5 & 0.0 & 1.0 & 0.5 & 0.5 & 0.5 & 0.0 & 0.0 & 1.0 \\ \bottomrule
    \end{tabular}
    \caption{Gewichtung der einzelnen Metriken zur Bestimmung des Input-Output-Overlap gesamt Scores}

\end{table}
  
\end{frame}
  \section{Evaluierung der Modelle}\raggedbottom
\label{evalmetric}
Die unterschiedlichen Optimierungen der Latentvektoren für ein Optimus Modell und des Cellstates für \textsc{BiMeanVAE} werden auf dem Amazon und dem Yelp Datensatz verglichen und mit dem aktuellem State-of-the-Art Modell COOP in Relation gesetzt.
Es existieren im Dev-Datensatz und Test-Datensatz jeweils 8 Eingabebewertungen und drei Gold-Summaries zum Evaluieren der generierten Ausgabebewertungen.
Die Eingabebewertungen werden durch ein Optimus VAE Modell und ein \textsc{BiMeanVAE} Modell in Latentvektoren umgewandelt.
Anschließend werden mittels in Abschnitt \ref{coop_chap} dargestellter COOP Herangehensweise die Latentvektoren kombiniert, um den \textit{Input-Output-Overlap} zu maximieren. 
Weiterhin werden die Latentvektoren mittels in dieser Arbeit eingeführtem Attribut-Modell optimiert, um detailreiche und umfassende Durchschnittsrezensionen zu erhalten.


\subsection{Evaluationsmetriken}
Die generierten Textbewertungen werden mit den drei Gold-Summaries verglichen.
Zum Vergleich der generierten Rezensionen wird die ROUGE (\textbf{R}ecall-\textbf{O}riented \textbf{U}nderstudy for \textbf{G}isting \textbf{E}valuation)-Metrik verwendet \citep{lin-2004-rouge}.
Die ROUGE-N Metrik misst die Anzahl der übereinstimmenden N-Grams zwischen dem generierten Text und den Referenztexten. 
Ein N-Gram ist eine N-lange sequentielle Folge von Wörtern innerhalb der Texte. 

Zur Bewertung der generierten Rezensionen werden die vorhergesagten Ergebnisse mit den korrekten Ergebnissen verglichen. 
Die Konfusionsmatrix in Abbildung \ref{confusionmatrix} ist eine Wahrheitsmatrix, welche die Einteilung der vorhergesagten Ergebnisse ermöglicht. 
True Positive (TP) und True Negative (TN) sind von dem Modell korrekt vorhergesagte Ergebnisse, False Positive (FP) und False Negative (FN) ist eine Klasse von falsch vorhergesagten Ergebnissen.



\begin{figure}[h!]
    \centering
\begin{tikzpicture}[
    box/.style={draw,rectangle,minimum size=2cm,text width=1.5cm,align=left}]
    \matrix (conmat) [row sep=.1cm,column sep=.1cm] {
    \node (tpos) [box,
        label=left:Positive,
        label=above:Positive,
        ] {True \\ positive};
    &
    \node (fneg) [box,
        label=above:Negative] {False \\ negative};
    \\
    \node (fpos) [box,
        label=left:Negative] {False \\ positive};
    &
    \node (tneg) [box] {True \\ negative};
    \\
    };
    \node [left=.05cm of conmat,text width=1.5cm,align=right] {\textbf{Referenz}};
    \node [above=.05cm of conmat] {\textbf{Vorhersage}};
\end{tikzpicture}
\caption{Konfusionsmatrix zur Berechnung des ROUGE-N Scores}
\label{confusionmatrix}
\end{figure}

Zur Berechnung des ROUGE-N Scores werden die einzelnen Textabschnitte in eine Menge aus N-Grams zerlegt.
Mittels der Konfusionsmatrix in Abbildung \ref{confusionmatrix} lassen sich Precision (P) und Recall (R) definieren:
\begin{addmargin}[30pt]{30pt}
    \textbf{Precision}: 
    Der Precision Wert ergibt sich aus dem Verhältnis der korrekt vorhergesagten N-Grams und der Anzahl der insgesamt vorhergesagten N-Grams.
    \begin{align*}
    \text{P} = \frac{\text{TP}}{\text{TP}+\text{FP}}
    \end{align*}

    \textbf{Recall}:
    Recall ist als Verhältnis zwischen den korrekt vorhergesagten N-Grams und den N-Grams aus der Referenz definiert.
    \begin{align*}
    \text{R} = \frac{\text{TP}}{\text{TP}+\text{FN}}
    \end{align*}

    $\textbf{F}_\textbf{1}$:
    Das F1-Maß beschreibt das harmonische Mittel zwischen Precision und Recall.
    \begin{align*}
    \text{F}_\text{1} = \frac{2\text{PR}}{\text{P}+\text{R}}
    \end{align*}
\end{addmargin}

In der Evaluation werden die ROUGE-1, ROUGE-2 und ROUGE-L Werte miteinander verglichen.
ROUGE-1 verwendet als N-Gram Unigramme, ROUGE-2 Bigramme und ROUGE-L misst die längste gleiche Subsequenz zwischen Vorhersage und Referenz.

Da die ROUGE-Scores lediglich die einzelnen Wortsequenzen miteinander vergleichen, findet die semantische Bedeutung und Ähnlichkeit der Bewertungen mit der Referenz keinen Einfluss.
Hier erzielen zum Beispiel Synonyme keine guten ROUGE-Scores, obwohl sie eine semantische Übereinstimmung haben.
Um trotzdem die semantische Ähnlichkeit zwischen Bewertungen und Referenz zu messen, wird als weitere Metrik der Moverscore aus Abschnitt \ref{moverscore} verwendet.
Der Moverscore basiert auf BERT und vergleicht Context-Embeddings mittels Earth-Mover-Distance \citep{emd}. Als Metrik konnte der Moverscore hohe Korrelationen mit menschlichem Urteilsvermögen aufweisen.


\subsection{Moverscore}
\label{moverscore}
Der Moverscore \citep{moverscore_paper} ist eine Evaluationsmetrik, die semantische Inhalte zwischen zwei Textsequenzen vergleicht und diesen einen Ähnlichkeitswert zuweist.
Das Ziel vom Moverscore ist es eine Metrik abzubilden, die einer menschlichen Bewertung der Ähnlichkeit von zwei Sequenzen am nähesten ist. 
Im Gegensatz zu anderen Textähnlichkeitsmetriken, die lediglich die Überlappungen von Tokens innerhalb der Sequenzen messen ohne die Semantik der Wörter zu bewerten, 
bildet sich der Moverscore aus einer Kombination bestehend aus einer im Kontext eingebetteten Repräsentation der einzelnen Textsequenzen, die eine semantische Distanz untereinander abbilden.
Die semantische Distanz wird über die Word Mover Distance \citep{wordmoverdistance}, einer Metrik basierend auf der Earth Mover Distance, bestimmt. Es wird ein minimaler Transportfluss zwischen den einzelnen Sequenzen errechnet.
Die Worteinbettungen werden durch ein BERT Modell erzeugt.

Insgesamt ist der Moverscore für die Bewertungsgenerierung ein wichtiger Leistungsindikator, da nicht nur übereinstimmende N-Gramme an Wörtern gemessen werden, sondern die Semantik der einzeln Wörter miteinbezogen wird. 
Da insbesondere in Bewertungen ähnliche Meinungen auf unterschiedliche Weise ausgedrückt werden können, bietet sich der Moverscore hier gut als Metrik an, um diese Übereinstimmungen zu finden, siehe Abschnitt \ref{moverscore_ranking}.


% \subsection{Bewertung der Datensätze}

% \subsubsection{Amazon-Datensatz}

% \subsubsection{Yelp-Datensatz}

\subsection{Ergebnisse}
\label{eval_results_chapter}
In Tabelle \ref{eval_results} ist die Performance der unterschiedlichen untersuchten und erstellten Modelle dargestellt.
In dieser Arbeit wurde das \glqq \textit{COOP}+Attribute Model\grqq{} Modell entwickelt.
Es basiert auf dem \textit{COOP} Modell und verbessert die Generierung von neuen Rezensionen durch die Verwendung eines Attribut-Modells.
Unterschieden werden die \textit{COOP} Modelle durch ihre grundlegend verwendete Variational Autoencoder Architektur in Optimus und \textsc{BiMeanVAE}, siehe Kapitel \ref{vae}.
Das Optimus Modell kombiniert BERT und GPT-2 in einem Variational Autoencoder Modell. \textsc{BiMeanVAE} hingegen besteht aus einem BiLSTM Encoder mit einem LSTM Decoder trainiert als Variational Autoencoder Modell.
Ebenfalls werden die weiteren aktuellen Modelle LexRank, MeanSum und CopyCat in dem Vergleich miteinbezogen.

Verglichen wird die Performance der unterschieldichen Modelle mit den zuvor beschriebenen Evaluationsmetriken, dem ROUGE-1, ROUGE-2, ROUGE-L Score und dem Moverscore.
Diese Metriken lassen ausreichend Rückschlüsse auf die erreichte Performance der Modelle und einer Leistungssteigerung zwischen den \textit{COOP} Basismodellen und den modifizierten \textit{COOP} mit Attributionsmodellen zu.
% \newcommand\crule[3][black]{\textcolor{#1}{\rule{#2}{#3}}}
% \crule[green!50!white!100]{2pt}{8pt}
\begin{table}[!h]
  
    \centering
    \begin{tabular}{@{}lcccc|cccc@{}}
    \toprule
             Test-Dataset                  & \multicolumn{4}{c}{Amazon} & \multicolumn{4}{c}{Yelp} \\ 
    \textbf{Method} & \textbf{R1} & \textbf{R2} & \textbf{RL} & \textbf{MV} & \textbf{R1} & \textbf{R2} & \textbf{RL} & \textbf{MV}\\ \midrule
    % \textit{COOP + Attribute Model - DEV Scores}        &         &         &        &        &        &   & &     \\
    % $\quad$ Optimus            &     \textbf{37.01}    &   \underline{7.44}  &  20.55  & \textbf{23.86} &   \textbf{35.99}   &   \textbf{7.79}       & \underline{19.40}   &   23.56 \\ 
    % $\quad$ \textsc{BiMeanVae}   &   \underline{36.47}   &   \textbf{7.59}    &   \textbf{22.22}  & 23.05 &     &      &   &    \\ \midrule
    
    \textit{COOP + Attribute Model}        &         &         &        &        &        &   & &     \\
    $\quad$ Optimus            &   35.68   & \textbf{7.55}  &  20.68 & \textbf{56.73} &  33.85   &  6.98  & 18.82  &  56.47  \\ 
    $\quad$ \textsc{BiMeanVae}  &   \textbf{37.94}  &   7.20    &  \textbf{21.75} & \underline{56.69} &   \underline{34.97}  & \underline{7.06}     & \underline{19.86}  &  \textbf{56.90}  \\ \midrule

    \textit{COOP}              &         &         &        &        &        & &   &    \\ %PAPER
    $\quad$ Optimus   $^{\star}$          & 35.32 &6.22 &19.84  & 56.41&  33.68& 7.00 &18.95 & 56.41\\ 
    $\quad$ \textsc{BiMeanVae}  $^{\star}$   & \underline{36.57} &\underline{7.23} &\underline{21.24} & 56.49 & \textbf{35.37} & \textbf{7.35} &\textbf{19.94} & \underline{56.78} \\ \midrule
    

    % \textit{COOP}  (real)            &         &         &        &        &        & &   &    \\
    % $\quad$ Optimus           & 35.32 &6.22 &19.84  & 56.41 & 33.60  & 7.00   & 18.95 & 56.41\\ 
    % $\quad$ \textsc{BiMeanVae}  & 36.40 &  7.16 &  21.08 & 56.49 & 35.37  & \underline{7.35}  & \textbf{19.94} & 56.78\\ \midrule %biamz check
    
    % \textit{COOP-DEV}              &         &         &        &        &        & &   &    \\
    % $\quad$ Optimus $^{\star}$           & 35.32   & 6.22    & 19.84 & \underline{23.22} & 33.60  & 7.00   & 18.95 & 23.33\\ 
    % $\quad$ \textsc{BiMeanVae}$^{\dagger}$  & $\text{35.67}^{\dagger}$    & $\text{6.53}^{\dagger}$   & \underline{$\text{21.07}^{\dagger}$} & 22.12 & \underline{35.37}  & \underline{7.35}  & \textbf{19.94} & 23.78\\ \midrule
    

    \textit{SimpleAvg}                   &       &      &       &       &       &      &       & \\
    $\quad$ Optimus  $^{\star}$          & 33.54 & 6.18 & 19.34 & 56.49 & 31.23 & 6.48 & 18.27 & 23.05\\
    $\quad$ \textsc{BiMeanVae}$^{\star}$ & 33.60 & 6.64 & 20.87 & 20.85 & 32.87 & 6.93 & 19.89 & 22.41\\
    $\quad$ CopyCat  $^{\star}$          & 31.97 & 5.81 & 20.16 & -     & 29.47 & 5.26 & 18.09 & -\\ 
    $\quad$ MeanSum  $^{\star}$          & 29.20 & 4.70 & 18.15 & -     & 28.46 & 3.66 & 15.57 & -\\ \midrule
    \textit{Extractive}                  &       &      &       &       &       &      &       &      \\
    $\quad$ LexRank  $^{\star}$          & 28.74 & 5.47 & 16.75 & -     & 25.01 & 3.62 & 14.67 & -\\ \bottomrule
    \end{tabular}
    \caption{ROUGE und Moverscore Ergebnisse auf den Test-Benchmarkdatensätzen der unterschiedlichen Modelle. Die besten Ergebnisse sind fett markiert und die zweitbesten Ergebnisse unterstrichen.
    Der Stern $^{\star}$ denotiert, dass die ROUGE-Ergebnisse aus den Ergebnissen von \citep{coop} übernommen wurden.
    %$^{\dagger}$ Evaluierte Performance unterscheidet sich vom \citep{coop} Paper. Performance wurde nach SourceCode des Papers bestimmt.
    }
    \label{eval_results}
\end{table}

Grundsätzlich lassen sich die Ergebnisse in Tabelle \ref{eval_results} in die Kategorien abstraktive und extraktive Zusammenfassung unterteilen.
Hier ist eindeutig zu erkennen, dass LexRank als extraktive Methode in allen Bereichen den abstraktiven Methoden unterliegt. 
Die Gruppe der abstraktiven Methoden umfasst die \textit{SimpleAVG} Gruppe, die \textit{COOP} Gruppe und die \textit{COOP+Attribute Model} Gruppe, die alle auf Variational Autoencodern basieren.
Diese Gruppen wurden nach der Kombinationsmethode der einzelnen Latentvektoren zu einem repräsentativen Latentvektor unterschieden.


Die \textit{SimpleAvg} Gruppe umfasst unterschiedliche abstraktive Textzusammenfassungsmethoden auf Basis von Variational Autoencodern.
Bei dieser Gruppe wird von allen erzeugten Latentvektoren ein normaler Durchschnittsvektor errechnet, von dem anschließend gesamplet wird.
Es ist erkennbar, dass die beiden Methoden Optimus und \textsc{BiMeanVAE} den Methoden CopyCat und MeanSum überlegen sind, da diese in allen Messwerten bessere Ergebnisse erzielen.
Besonders hervozuheben ist hier, dass Optimus bessere Moverscore Ergebnisse als \textsc{BiMeanVAE} erzielt, allerdings in den ROUGE Werten minimal schlechter abschneidet. 

Eine große Leistungssteigerung ergibt sich durch die \textit{COOP} Methode, die eine optimale Kombination der einzelnen Latentvektoren findet.
Hier erzielen sowohl Optimus als auch \textsc{BiMeanVAE} in allen Metriken bessere Ergebnisse als die \textit{SimpleAvg} Vergleichsgruppe.
Demnach ist das Durchsuchen der Kombinationen von Latentvektoren sinnvoll. 
Insbesondere der ROUGE-1 Score übertrifft die \textit{SimpleAVG} Scores bei Optimus und \textsc{BiMeanVAE} signifikant im Durchschnitt um 1.93\%.  %PROZENT COOP-SIMPLEAVG
Die größte Leistungssteigerung zwischen der \textit{COOP} Kombinationsstrategie und \textit{SimpleAvg} erfährt \textsc{BiMeanVAE}.
Dies ist sehr beeindruckend, da \textsc{BiMeanVAE} mit 13 Millionen Parametern weitaus weniger Parameter hat als Optimus mit 239 Millionen Parametern und auch nicht auf vortrainierte Sprachmodelle zurückgreifen kann.
Demnach lassen sich mittels Variational Autoencoder Textsequenzen hervorragend in Latentvektoren encodieren und diese mittels Vektoroperationen kombinieren.


%Vergleich Optimus vs BiMeanVAE an Texten
Eine weitere Leistungssteigerung der verwendeten Variational Autoencoder Optimus und \textsc{BiMeanVAE} mit \textit{COOP} Kombinatorik lässt sich durch die in dieser Arbeit verwendeten Attributmodelle feststellen.
Die aus dem Latentvektoren decodierten Textsequenzen lassen sich so noch stärker in die gewünschte Richtung bei der Generierung adaptieren. 
Das verwendete Attributmodell ist ein Bag of Words Modell, welches aus den 150 am häufigsten vorkommenden Tokens besteht. 
Somit wird die Gewichtung bei der Generierung auf die häufig vorkommenden Tokens gelenkt und die generierten Textsequenzen haben eine höhere Überlappung.
Beispielsweise fällt auf, dass beim Generieren teilweise mehrere vorgeschlagene Tokens ein hohes Ranking erhalten, wobei am Ende durch das Attributionsmodell das am besten passende mit einer höheren Wahrscheinlichkeit ausgewählt wird.
Insbesondere spezifische Begriffe die ausschlaggebend für die erzeugten Rezensionen sind, allerdings in einer normalen Sprachverteilung eine geringe Gewichtung erhalten würden, werden durch das Bag of Words Attributmodell stark hervorgehoben.

Die Performance des kombinierten \textit{COOP + Attributmodells} zeigt insbesondere auf dem Amazon Datensatz im Optimus sowie auch \textsc{BiMeanVAE} Modell deutliche Verbesserungen gegenüber dem COOP Modellen.
Die ROUGE Scores, sowie auch der Moverscore übertrifft für das jeweilige Modell mit Attributmodell die Scores des unoptimierten COOP Modells. 
Demnach konnte erfolgreich mittels Attributmodell die Generierung auf dem Amazon Datensatz optimiert werden.

Auf dem Yelp Datensatz zeigt sich keine deutliche Leistungssteigerung durch Verwendung des Attributmodells. 
Die erreichten Scores sind ähnlich wie die des unoptimierten COOP Modells. 
Das \textsc{BiMeanVAE} + Attributmodell erreicht minimal schlechtere Scores als das Baseline COOP Modell.
In den Moverscores und den ROUGE-1 Score des Optimus Modell zeigt sich eine minimale Leistungssteigerung.
Die fehlende Leistungssteigerung in den Metriken ist mit der Aufbau des Yelp Datensatzes erklärbar. 
Der Yelp Datensatz hat jeweils nur eine einzige Referenzrezension an der die Scores errechnet werden.
Somit können durchaus Bewertungen, die semantisch den Inhalt perfekt wiedergeben, äußerst schlechte Scores erhalten, da keine überlappenden N-Gramme zu finden sind.

Um die Optimierungen des Attributmodells weiter zu untersuchen werden in Abschnitt \ref{oracle} die Ergebnisse, die durch Auswahl der finalen Rezension aus den Rezensionen durch ein Orakel entstehen, ausgewertet.
Somit lässt sich der Einfluss der Rankingfunktion auf die Ergebnisse ausschließen.
Anschließend werden einzelne Rezensionen aus den beiden Datensätzen in Abschnitt \ref{example} betrachtet und untereinander verglichen.



\subsection{Vergleich der Modelle mit Rezensionsauswahl durch Orakel}
\label{oracle}
Zum Vergleich von COOP + Attributmodell mit dem COOP Modell in Bezug auf die reine Generationsleistung, wurden die Rezensionen mittels Orakel ausgewählt, um den Einfluss der Rankingfunktion zu unterdrücken.
Bei der Auswertung in Abschnitt \ref{eval_results_chapter} wurden die Rezensionen anhand einer Rankingfunktion in Bezug zu den Eingaberezensionen ausgewählt.
In diesem Vergleich werden die Rezensionen anhand eines Orakels, welches die Rezensionen, die mit den Gold-Zusammenfassungen die größte ROUGE-1 Übereinstimmung haben, ausgewählt werden.
Somit sind die Ergebnisse in diesem Vergleich gleichzeitig eine obere Schranke für die Leistungsfähigkeit der entsprechenden Modelle.


\begin{table}[!h]
    \centering
    \begin{tabular}{@{}lcccc|cccc@{}}
    \toprule
             Test-Dataset                  & \multicolumn{4}{c}{Amazon} & \multicolumn{4}{c}{Yelp} \\ 
    \textbf{Method} & \textbf{R1} & \textbf{R2} & \textbf{RL} & \textbf{MV} & \textbf{R1} & \textbf{R2} & \textbf{RL} & \textbf{MV}\\ \midrule
      
    \textit{COOP + Attribute Model}        &         &         &        &        &        &   & &     \\
    $\quad$ Optimus                         & 40.75 & \underline{8.82} & 21.54 & 56.88 & \underline{43.81} & \underline{11.08} & 22.43 & 57.34 \\ 
    $\quad$ \textsc{BiMeanVae}              & \textbf{42.15} & 8.67 & \textbf{23.52} & \textbf{57.18} & \textbf{45.13} & \textbf{11.23} & \textbf{24.77} & \textbf{57.82} \\ \midrule
    
    \textit{COOP}                           &       &      &       &       &       &       &       &        \\ %PAPER
    $\quad$ Optimus                          & 39.84 & 7.88 & 21.40 & \underline{57.02} & 42.05 & 10.03 & 22.24 & 57.32\\ 
    $\quad$ \textsc{BiMeanVae}               & \underline{40.80} & \textbf{8.99} & \underline{23.48} & 56.83 & 42.72 & 10.21 & \underline{24.00} & \underline{57.44} \\ \bottomrule
    \end{tabular}
    \caption{ROUGE und Moverscore Ergebnisse auf den Test-Benchmarkdatensätzen der unterschiedlichen Modelle mit Auswahl der generierten Rezensionen durch ein Orakel. Die besten Ergebnisse sind fett markiert und die zweitbesten Ergebnisse unterstrichen.}
    \label{oracle_results}
\end{table}

In Tabelle \ref{oracle_results} sind die Ergebnisse für die COOP + Attributionsmodell Modelle und die COOP Modelle mit Rezensionsauswahl durch ein Orakel dargestellt.
In allen Metriken sind hier die COOP + Attributionsmodell Ergebnisse den COOP Modell Ergebnissen überlegen. 
Die Ergebnisse des \textsc{BiMeanVAE} + Attributionsmodell Modells übertreffen die Ergebnisse des COOP Modells auf beiden Datensätzen in allen Metriken stark, bis auf den ROUGE-2 Wert des Amazon Datensatzes.
Das Optimus + Attributionsmodell Modell übertrifft ebenfalls die Ergebnisse des COOP Modells, außer dem Moverscore auf dem Amazon und den ROUGE-L Score auf dem Yelp Datensatz.

Somit lässt sich eine starke Leistungssteigerung durch Verwendung des Attributmodells feststellen. 
Das Attributmodell ermöglicht bei der Generierung die Gewichtung von Tokens im Bag Of Words des Attributmodells zu forcieren, wodurch wiederum die exakten Wörter der Eingaberezensionen verwendet werden.
Die Metriken lassen ebenfalls Rückschlüsse auf eine bessere Anpassungsfähigkeit durch das Attributmodell zu, da in fast allen Metriken das Referenzmodell übertroffen wurde.
Insbesondere auch der Moverscore, der die semantische Ähnlichkeit abbildet, zeigt eine Leistungssteigerung durch das Attributmodell.
Zwischen den beiden Modellen Optimus und \textsc{BiMeanVAE} ist \textsc{BiMeanVAE} mit Attributionsmodel das Modell mit den besten Ergebnissen.

\subsection{Beispiele für generierte Rezensionen}
\label{example}


\small
%Define a reference depth. 
%You can choose either relative or absolute.
%--------------------------
\newlength{\DepthReference}
\settodepth{\DepthReference}{g}%relative to a depth of a letter.
\setlength{\DepthReference}{1pt}%absolute value.

%Define a reference Height. 
%You can choose either relative or absolute.
%--------------------------
\newlength{\HeightReference}
\settoheight{\HeightReference}{T}
\setlength{\HeightReference}{7pt}


%--------------------------
\newlength{\Width}%

\newcommand{\ccolorbox}[2][red]%
{%
    \settowidth{\Width}{#2}%
    \setlength{\fboxsep}{1pt}%
    \colorbox{#1}%
    {%      
        \raisebox{-\DepthReference}%
        {%
                \parbox[b][\HeightReference+\DepthReference][c]{\Width}{\centering#2}%
        }%
    }%
}



\definecolor{HighlightColor}{HTML}{dc2626}
\definecolor{BackgroundColor}{HTML}{bfdbfe}
\normalsize


Folgend werden einige generierte Textrezensionen des COOP Modells und des COOP + Attributionsmodell Modells direkt miteinander verglichen.
Es wurden jeweils zwei generierte Rezensionen des COOP und des COOP + Attributionsmodell Modells zu den Datensätzen Amazon und Yelp ausgewählt. 
Zur besseren Darstellung sind bei den generierten Rezensionen die mit der Gold-Zusammenfassung übereinstimmenden Unigrams \textcolor{HighlightColor}{rot} markiert, Bigrams sind \ccolorbox[BackgroundColor]{farblich blau hinterlegt} und die längste übereinstimmende Textsequenz \underline{unterstrichen}.
Der Moverscore lässt sich nicht visualisieren.

Authentische Rezensionen sollten konsistent in ihrem Inhalt sein, präzise auf Eigenschaften und Aspekte der Produkte oder Dienstleistungen eingehen und der Inhalt der generierten Rezension sollte mit den Eingabebewertungen deckungsgleich sein.
Anhand dieser Kriterien werden nachfolgend mehrere generierte Rezensionen bewertet.

% \setlength{\DepthReference}{6pt}
% \setlength{\HeightReference}{6pt}

\setlength{\fboxsep}{0.7em}

\subsubsection{Amazon Rezensionen}
%ROUGE-L MARKIEREN Überall

Nachfolgend werden zwei generierte Rezensionen des Amazon Datensatzes evaluiert. Zu den Amazon Pordukten existieren jeweils drei Referenzzusammenfassungen, wodurch eine bessere Vergleichbarkeit mittels Metriken möglich ist.

\begin{Rezension}[!h]
    \centering
    %\scriptsize
    \small
    \framebox{
        \parbox{\columnwidth-4\fboxsep}{
            \includegraphics[width=1.3cm]{bilder/necklace.jpg} \textbf{Produkt:} Sterling Silver \glqq{}Love\grqq{} Open Heart Pendant Necklace, 18\grqq{} \\ \\
        \textbf{\textsc{BiMeanVAE} COOP+Attribute-Model:} \ccolorbox[BackgroundColor]{ \textcolor{HighlightColor}{\strut This necklace}} \ccolorbox[BackgroundColor]{ \textcolor{HighlightColor}{\strut is a}} \ccolorbox[BackgroundColor]{ \textcolor{HighlightColor}{\strut great quality}} \textcolor{HighlightColor}{and} \textcolor{HighlightColor}{so} \textcolor{HighlightColor}{far} \textcolor{HighlightColor}{it}\textcolor{HighlightColor}{'s} \textcolor{HighlightColor}{the} perfect \textcolor{HighlightColor}{size}\ccolorbox[BackgroundColor]{\textcolor{HighlightColor}{\strut . It}} \ccolorbox[BackgroundColor]{ \textcolor{HighlightColor}{\strut is a}} \textcolor{HighlightColor}{very} \textcolor{HighlightColor}{nice} \textcolor{HighlightColor}{looking} \textcolor{HighlightColor}{necklace}\textcolor{HighlightColor}{,} \ccolorbox[BackgroundColor]{ \textcolor{HighlightColor}{\strut and the}} color \underline{\ccolorbox[BackgroundColor]{ \textcolor{HighlightColor}{\strut is beautiful}}\ccolorbox[BackgroundColor]{ \textcolor{HighlightColor}{\strut. The}} \ccolorbox[BackgroundColor]{ \textcolor{HighlightColor}{\strut chain is}} \textcolor{HighlightColor}{a}} perfect gift \ccolorbox[BackgroundColor]{ \textcolor{HighlightColor}{\strut for the}} \textcolor{HighlightColor}{price} \textcolor{HighlightColor}{of} \textcolor{HighlightColor}{the} necklace. I love \textcolor{HighlightColor}{it}\textcolor{HighlightColor}{!} \\ 
        \textbf{Scores:} R-1: 48.63, R-2: 16.78, R-L: 28.08, MV: 60.05\\ \\
        %\textbf{\textsc{BiMeanVAE} COOP:} \textcolor{HighlightColor}{This} \ccolorbox[BackgroundColor]{ \textcolor{HighlightColor}{\strut is a}} \textcolor{HighlightColor}{great} product \underline{\ccolorbox[BackgroundColor]{ \textcolor{HighlightColor}{\strut for the}} \ccolorbox[BackgroundColor]{ \textcolor{HighlightColor}{\strut price .}}} \textcolor{HighlightColor}{It} \ccolorbox[BackgroundColor]{ \textcolor{HighlightColor}{\strut is a}} \textcolor{HighlightColor}{very} \textcolor{HighlightColor}{good} \textcolor{HighlightColor}{quality} \textcolor{HighlightColor}{,} \ccolorbox[BackgroundColor]{ \textcolor{HighlightColor}{\strut and the}} \textcolor{HighlightColor}{price} was right \ccolorbox[BackgroundColor]{ \textcolor{HighlightColor}{\strut . The}} only thing \textcolor{HighlightColor}{is} that \ccolorbox[BackgroundColor]{ \textcolor{HighlightColor}{\strut it is}} \textcolor{HighlightColor}{a} little small \ccolorbox[BackgroundColor]{ \textcolor{HighlightColor}{\strut , but}} \textcolor{HighlightColor}{it} \textcolor{HighlightColor}{'s} not too big \ccolorbox[BackgroundColor]{ \textcolor{HighlightColor}{\strut . It}} \textcolor{HighlightColor}{'s} \ccolorbox[BackgroundColor]{ \textcolor{HighlightColor}{\strut a great}} buy \ccolorbox[BackgroundColor]{ \textcolor{HighlightColor}{\strut for the}} \ccolorbox[BackgroundColor]{ \textcolor{HighlightColor}{\strut price .}}  \\ 
        %\textbf{Scores:} R-1: 36.61, R-2: 8.30, R-L: 26.44, MV: 55.57 \\ \\
        \textbf{\textsc{BiMeanVAE} COOP:} \textcolor{HighlightColor}{It} was \textcolor{HighlightColor}{a} perfect gift \ccolorbox[BackgroundColor]{ \textcolor{HighlightColor}{for a}} friend \textcolor{HighlightColor}{of} her birthday\textcolor{HighlightColor}{.} She loves \textcolor{HighlightColor}{it}\textcolor{HighlightColor}{,} \textcolor{HighlightColor}{the} \textcolor{HighlightColor}{look} \textcolor{HighlightColor}{is} \textcolor{HighlightColor}{great}\textcolor{HighlightColor}{,} \ccolorbox[BackgroundColor]{ \textcolor{HighlightColor}{and the}} \textcolor{HighlightColor}{price} was \textcolor{HighlightColor}{great}\ccolorbox[BackgroundColor]{\textcolor{HighlightColor}{. It}} was easy \textcolor{HighlightColor}{to} set up\textcolor{HighlightColor}{,} \textcolor{HighlightColor}{and} she loves it. I \underline{\ccolorbox[BackgroundColor]{ \textcolor{HighlightColor}{would recommend}} \textcolor{HighlightColor}{this}} product \ccolorbox[BackgroundColor]{ \textcolor{HighlightColor}{to anyone}}\textcolor{HighlightColor}{.}   \\ 
        \textbf{Scores:} R-1: 33.56, R-2: 5.71, R-L: 20.27, MV: 56.00 \\ \\
    

        \textbf{Optimus COOP+Attribute-Model:} \textcolor{HighlightColor}{This} \ccolorbox[BackgroundColor]{ \textcolor{HighlightColor}{\strut is a}} \textcolor{HighlightColor}{great} gift \underline{\ccolorbox[BackgroundColor]{ \textcolor{HighlightColor}{\strut for the}} \ccolorbox[BackgroundColor]{ \textcolor{HighlightColor}{\strut price.}}} She loves \textcolor{HighlightColor}{it}\textcolor{HighlightColor}{,} \textcolor{HighlightColor}{so} much better than \textcolor{HighlightColor}{the} \textcolor{HighlightColor}{necklace} \textcolor{HighlightColor}{and} \textcolor{HighlightColor}{it} looks \ccolorbox[BackgroundColor]{ \textcolor{HighlightColor}{\strut beautiful.}} You \textcolor{HighlightColor}{can't} \textcolor{HighlightColor}{get} \textcolor{HighlightColor}{a} gift on \ccolorbox[BackgroundColor]{ \textcolor{HighlightColor}{\strut the chain}} \ccolorbox[BackgroundColor]{ \textcolor{HighlightColor}{\strut, but}} \ccolorbox[BackgroundColor]{ \textcolor{HighlightColor}{\strut it is}} not worth \textcolor{HighlightColor}{the} money\textcolor{HighlightColor}{!} \textcolor{HighlightColor}{It} looks \textcolor{HighlightColor}{great}\textcolor{HighlightColor}{!}         \\ 
        \textbf{Scores:} R-1: 42.65, R-2: 9.28, R-L: 26.57, MV: 58.31\\ \\
        % \textbf{Optimus COOP:}  \ So \textcolor{HighlightColor}{far} \textcolor{HighlightColor}{the} \underline{\ccolorbox[BackgroundColor]{ \textcolor{HighlightColor}{\strut necklace is}} \textcolor{HighlightColor}{beautiful}} \textcolor{HighlightColor}{,} \ccolorbox[BackgroundColor]{ \textcolor{HighlightColor}{\strut and the}} perfect gift \textcolor{HighlightColor}{for} someone who \ccolorbox[BackgroundColor]{ \textcolor{HighlightColor}{\strut is a}} \ccolorbox[BackgroundColor]{ \textcolor{HighlightColor}{\strut beautiful piece}} \ccolorbox[BackgroundColor]{ \textcolor{HighlightColor}{\strut . It}} looks \textcolor{HighlightColor}{great} on \textcolor{HighlightColor}{the} picture \ccolorbox[BackgroundColor]{ \textcolor{HighlightColor}{\strut , but}} \textcolor{HighlightColor}{it} does not \textcolor{HighlightColor}{look} like \textcolor{HighlightColor}{a} gift \textcolor{HighlightColor}{.} Im \textcolor{HighlightColor}{very} happy with \textcolor{HighlightColor}{this} \textcolor{HighlightColor}{necklace} \textcolor{HighlightColor}{and} \ccolorbox[BackgroundColor]{ \textcolor{HighlightColor}{\strut it is}} not worth \textcolor{HighlightColor}{the} money \textcolor{HighlightColor}{!}  \\ 
        % \textbf{Scores:} R-1: 40.26, R-2: 13.01, R-L: 23.48, MV: 58.06 }
        \textbf{Optimus COOP:}  Its \ccolorbox[BackgroundColor]{ \textcolor{HighlightColor}{a beautiful}} \textcolor{HighlightColor}{necklace}\textcolor{HighlightColor}{,} \textcolor{HighlightColor}{and} \textcolor{HighlightColor}{it} looks \textcolor{HighlightColor}{great} \underline{\ccolorbox[BackgroundColor]{ \textcolor{HighlightColor}{for the}} \ccolorbox[BackgroundColor]{ \textcolor{HighlightColor}{price.}}} She said \textcolor{HighlightColor}{it} was \textcolor{HighlightColor}{a} gift \textcolor{HighlightColor}{for} her birthday present\ccolorbox[BackgroundColor]{\textcolor{HighlightColor}{, but}} \ccolorbox[BackgroundColor]{ \textcolor{HighlightColor}{it is}} \textcolor{HighlightColor}{very} \textcolor{HighlightColor}{thin} \textcolor{HighlightColor}{and} not worth \textcolor{HighlightColor}{the} money\textcolor{HighlightColor}{.} If \textcolor{HighlightColor}{you} want \textcolor{HighlightColor}{to} go wrong with \textcolor{HighlightColor}{this} \textcolor{HighlightColor}{necklace}\textcolor{HighlightColor}{!}  \\ 
        \textbf{Scores:} R-1: 37.76, R-2: 8.57, R-L: 21.67, MV: 57.69 }
    
        }
    \caption{Vergleich der generierten Rezensionen zwischen dem COOP und COOP + Attributionsmodell zu Produkt B0040EIHQQ des Amazon Datensatzes}
\label{reviewAmz1}
\end{Rezension}

Die in Rezension \ref{reviewAmz1} generierten Rezensionen für eine Halskette des Amazons Datensatzes zeigen insgesamt gute Ergebnisse.
Der Inhalt der beiden durch \textsc{BiMeanVAE} generierten Rezensionen ist konsistent. 
Das Sentiment ist positiv und die Kette wird abschließend zum Kaufen empfohlen. Besonders auffällig ist die höhere Präzision des Attributmodells. Das Produkt und die Eigenschaften dieses werden explizit erwähnt, wie zum Beispiel \glqq{}great quality\grqq{}, \glqq{}perfect size\grqq{}, \glqq{}very nice looking necklace\grqq{}, \glqq{}color is beautiful\grqq{} und \glqq{}perfect gift for the price\grqq{}.
Im Gegensatz dazu ist die vom COOP Modell generierte Rezension allgemeiner und erwähnt lediglich das gute Aussehen der Halskette \glqq{}the look is great\grqq{} und den Preis \glqq{}price was great\grqq{} des Produktes ohne auf spezifische Eigenschaften des Produktes wie zum Beispiel die Farbe einzugehen.
Diese Auswertung spiegelt sich auch in den Metriken wieder, in denen das Attributmodell die Metriken des COOP Modells in allen Werten übertrifft. 
Insbesondere der Moverscore, der die semantische Ähnlichkeit zu den Gold-Zusammenfassungen angibt, ist mit 60.05 beim Attributmodell bedeutend höher als die 56.00 des COOP Modells.
In dieser Beispielrezension bestätigt sich die Hypothese, dass das Attributmodell durch geeignete Tokens in einem Bag of Words Modell die Generierung präzise konditionieren können.

Die durch Optimus erstellten Rezensionen sind sowohl beim COOP, wie auch beim Attributmodell inkonsistent in ihrem Inhalt. 
Die Rezensionen haben beide hauptsächlich ein positives Sentiment, widersprechen sich allerdings beide jeweils mit den Formulierungen \glqq{}great for the price\grqq{} und \glqq{}not worth the money\grqq{}.
Trotzdem gehen beide Modelle auf unterschiedliche Aspekte der zu bewertenden Halskette ein und beschreiben den Stil der Kette als \glqq{}it looks beautiful\grqq{},\glqq{}It looks great!\grqq{} und \glqq{}a beautiful necklace\grqq{}.
Eine Steigerung der Leistung der generierten Rezensionen des Attributmodells gegenüber dem COOP Modell kann hier nur an den Metriken festgestellt werden und spiegelt sich nur minimal in der generierten Rezension wieder.
Hier fällt auf, dass das Attributmodell versucht die Kette \glqq{}the chain\grqq{} der Halskette anzusprechen, allerdings keinen semantisch korrekten Satz mit diesen Tokens bildet.

\begin{Rezension}[!h]
    \centering
    %\scriptsize
    \small
    \framebox{
        \parbox{\columnwidth-4\fboxsep}{
            \includegraphics[width=1.5cm]{bilder/table.jpg} \textbf{Produkt:} Coaster CO-150430 5 Pc Dining Set, Chestnut \\ \\
        \textbf{\textsc{BiMeanVAE} COOP+Attribute-Model:} \textcolor{HighlightColor}{The} table \ccolorbox[BackgroundColor]{ \textcolor{HighlightColor}{is very}} \textcolor{HighlightColor}{nice} \textcolor{HighlightColor}{and} \textcolor{HighlightColor}{sturdy}\ccolorbox[BackgroundColor]{\textcolor{HighlightColor}{. It}} \textcolor{HighlightColor}{was} \ccolorbox[BackgroundColor]{ \textcolor{HighlightColor}{easy to}} \ccolorbox[BackgroundColor]{ \textcolor{HighlightColor}{put together}} \textcolor{HighlightColor}{and} \textcolor{HighlightColor}{the} \textcolor{HighlightColor}{color} \textcolor{HighlightColor}{is} perfect\ccolorbox[BackgroundColor]{\textcolor{HighlightColor}{. The}} \textcolor{HighlightColor}{chairs} \textcolor{HighlightColor}{are} \textcolor{HighlightColor}{a} bit \textcolor{HighlightColor}{small} \textcolor{HighlightColor}{for} \textcolor{HighlightColor}{the} price\ccolorbox[BackgroundColor]{\textcolor{HighlightColor}{, but}} \textcolor{HighlightColor}{it}'s \textcolor{HighlightColor}{not} \textcolor{HighlightColor}{a} big \textcolor{HighlightColor}{deal}\ccolorbox[BackgroundColor]{\textcolor{HighlightColor}{. It}} \textcolor{HighlightColor}{was} \ccolorbox[BackgroundColor]{ \textcolor{HighlightColor}{very easy}} \ccolorbox[BackgroundColor]{ \textcolor{HighlightColor}{to assemble}}\textcolor{HighlightColor}{.}        \\ 
        \textbf{Scores:} R-1: 42.75, R-2: 7.03, R-L: 23.66, MV: 54.49\\ \\
        \textbf{\textsc{BiMeanVAE} COOP:} \textcolor{HighlightColor}{The} table \ccolorbox[BackgroundColor]{ \textcolor{HighlightColor}{is very}} \textcolor{HighlightColor}{nice} \textcolor{HighlightColor}{and} \textcolor{HighlightColor}{sturdy}\ccolorbox[BackgroundColor]{\textcolor{HighlightColor}{. The}} \textcolor{HighlightColor}{chairs} \textcolor{HighlightColor}{are} \ccolorbox[BackgroundColor]{ \textcolor{HighlightColor}{easy to}} \ccolorbox[BackgroundColor]{ \textcolor{HighlightColor}{put together}} \textcolor{HighlightColor}{and} \textcolor{HighlightColor}{the} \textcolor{HighlightColor}{color} \textcolor{HighlightColor}{is} \textcolor{HighlightColor}{great}\ccolorbox[BackgroundColor]{\textcolor{HighlightColor}{. It}} \textcolor{HighlightColor}{was} \ccolorbox[BackgroundColor]{ \textcolor{HighlightColor}{easy to}} \textcolor{HighlightColor}{assemble}\textcolor{HighlightColor}{,} \textcolor{HighlightColor}{and} \textcolor{HighlightColor}{the} price \textcolor{HighlightColor}{was} right\ccolorbox[BackgroundColor]{\textcolor{HighlightColor}{. The}} \textcolor{HighlightColor}{size} \textcolor{HighlightColor}{is} perfect \ccolorbox[BackgroundColor]{ \textcolor{HighlightColor}{for a}} \textcolor{HighlightColor}{small} kitchen\textcolor{HighlightColor}{,} \textcolor{HighlightColor}{so} \textcolor{HighlightColor}{it}'s \textcolor{HighlightColor}{not} \textcolor{HighlightColor}{a} big \textcolor{HighlightColor}{deal}\textcolor{HighlightColor}{.}         \\ 
        \textbf{Scores:} R-1: 41.60, R-2: 9.70, R-L: 24.81, MV: 54.53 \\ \\

        \textbf{Optimus COOP+Attribute-Model:} Just received this table \textcolor{HighlightColor}{top} \textcolor{HighlightColor}{and} \textcolor{HighlightColor}{sturdy}\ccolorbox[BackgroundColor]{\textcolor{HighlightColor}{. The}} table \textcolor{HighlightColor}{is} perfect \textcolor{HighlightColor}{for} \textcolor{HighlightColor}{the} price\textcolor{HighlightColor}{,} \textcolor{HighlightColor}{so} \textcolor{HighlightColor}{it} \textcolor{HighlightColor}{was} \ccolorbox[BackgroundColor]{ \textcolor{HighlightColor}{easy to}}\ccolorbox[BackgroundColor]{\textcolor{HighlightColor}{assemble.}} One \ccolorbox[BackgroundColor]{ \textcolor{HighlightColor}{of the}} \textcolor{HighlightColor}{chairs} \textcolor{HighlightColor}{are} \textcolor{HighlightColor}{not} \ccolorbox[BackgroundColor]{ \textcolor{HighlightColor}{sturdy,}} \textcolor{HighlightColor}{sturdy} \textcolor{HighlightColor}{and} \textcolor{HighlightColor}{the} table \textcolor{HighlightColor}{was} \textcolor{HighlightColor}{not} worth \textcolor{HighlightColor}{the} money\textcolor{HighlightColor}{.} Other than \ccolorbox[BackgroundColor]{ \textcolor{HighlightColor}{that it}} \textcolor{HighlightColor}{is} \textcolor{HighlightColor}{a} \textcolor{HighlightColor}{great} \ccolorbox[BackgroundColor]{ \textcolor{HighlightColor}{purchase.}}         \\ 
        \textbf{Scores:} R-1: 32.85, R-2: 5.22, R-L: 20.44, MV: 53.52 \\ \\
        \textbf{Optimus COOP:} Just received this table \textcolor{HighlightColor}{and} \textcolor{HighlightColor}{sturdy}\ccolorbox[BackgroundColor]{\textcolor{HighlightColor}{. The}} table \textcolor{HighlightColor}{is} perfect \textcolor{HighlightColor}{for} \textcolor{HighlightColor}{the} price\textcolor{HighlightColor}{,} \ccolorbox[BackgroundColor]{ \textcolor{HighlightColor}{and it}} \textcolor{HighlightColor}{was} \ccolorbox[BackgroundColor]{ \textcolor{HighlightColor}{easy to}} \ccolorbox[BackgroundColor]{ \textcolor{HighlightColor}{assemble.}} One \ccolorbox[BackgroundColor]{ \textcolor{HighlightColor}{of the}} \textcolor{HighlightColor}{chairs} \textcolor{HighlightColor}{are} \textcolor{HighlightColor}{not} perfect\ccolorbox[BackgroundColor]{\textcolor{HighlightColor}{. It}} \textcolor{HighlightColor}{is} \textcolor{HighlightColor}{sturdy} \textcolor{HighlightColor}{and} \textcolor{HighlightColor}{not} worth \textcolor{HighlightColor}{the} money\textcolor{HighlightColor}{.}         \\ 
        \textbf{Scores:} R-1: 30.32, R-2: 6.72, R-L: 19.67, MV: 53.49 }
    }
    \caption{Vergleich der generierten Rezensionen zwischen dem COOP und COOP+Attributmodell zu Produkt B001EQJ5AU des Amazon Datensatzes}
    \label{reviewAmz2}
\end{Rezension}

Rezension \ref{reviewAmz2} bewertet ein Set bestehend aus einem Tisch und 4 Stühlen.
Die durch \textsc{BiMeanVAE} generierten Rezensionen sind in ihrem Inhalt konsistent und beschreiben präzise den Inhalt des Sets.
Beide Rezensionen erwähnen die Eigenschaften \glqq{}easy to put together\grqq{}, \glqq{}easy to assemble\grqq{}, die Farbe und die Qualität des Sets.
Somit werden die einzelnen Aspekte des Produktes in den beiden Rezensionen aufgegriffen.
Anschaulich ist die Optimierung des Attributmodells die Phrase \glqq{}it's not a big deal\grqq{} semantisch korrekt zu Integrieren und den Bezug zur Größe der Stühle zu setzen.
Das COOP Modell widerspricht sich hier in der Semantik mit \glqq{}The size is perfect for a small kitchen, so it's not a big deal\grqq{}, da kein negativer Punkt angesprochen wird.
In den Metriken erzielen beide Modelle gute Ergebnisse, wobei das Attributmodell einen höheren ROUGE-1 Score hat und das COOP Modell in den anderen Metriken bessere Ergebnisse erzielt.

Die durch Optimus erzeugten Rezensionen beschreiben ebenfalls den Inhalt des Sets und erwähnen die Eigenschaften \glqq{}easy to assemble\grqq{} und die Qualität.
Teilweise sind die erzeugten Sätze syntaktisch nicht korrekt. 
In diesem Beispiel ist erkennbar, dass das Attributmodell höchstwahrscheinlich ab der Stelle \glqq{}One of the chairs\grqq{} die Tokens bei der Generierung stakr beeinflusst hat und somit die Semantik des verbleibenden Teils positiv beeinflusst hat.
In den Metriken erzielt das Attributmodell in allen Metriken, außer dem ROUGE-2 Score, bessere Ergebnisse.

% \begin{Rezension}[!h]
%     \centering
%     %\scriptsize
%     \small
%     \framebox{
%         \parbox{\columnwidth-4\fboxsep}{
%             \includegraphics[width=1.5cm]{bilder/polarfit.jpg} \textbf{Produkt:} Polar FT2 Heart Rate Monitor \\ \\
%         \textbf{\textsc{BiMeanVAE} COOP+Attribute-Model:} If \ccolorbox[BackgroundColor]{ \textcolor{HighlightColor}{\strut you are}} looking \textcolor{HighlightColor}{for} \underline{\ccolorbox[BackgroundColor]{ \textcolor{HighlightColor}{\strut a heart}} \ccolorbox[BackgroundColor]{ \textcolor{HighlightColor}{\strut rate monitor}}} \textcolor{HighlightColor}{this} \ccolorbox[BackgroundColor]{ \textcolor{HighlightColor}{\strut is the}} best \ccolorbox[BackgroundColor]{ \textcolor{HighlightColor}{\strut . It}} 's \ccolorbox[BackgroundColor]{ \textcolor{HighlightColor}{\strut easy to}} \textcolor{HighlightColor}{use} \textcolor{HighlightColor}{,} \textcolor{HighlightColor}{and} \textcolor{HighlightColor}{it} works great \ccolorbox[BackgroundColor]{ \textcolor{HighlightColor}{\strut . The}} \textcolor{HighlightColor}{only} downside \textcolor{HighlightColor}{is} \textcolor{HighlightColor}{that} \textcolor{HighlightColor}{you} have \textcolor{HighlightColor}{to} \textcolor{HighlightColor}{be} careful \textcolor{HighlightColor}{not} \textcolor{HighlightColor}{to} take \textcolor{HighlightColor}{it} \ccolorbox[BackgroundColor]{ \textcolor{HighlightColor}{\strut out of}} \textcolor{HighlightColor}{the} \textcolor{HighlightColor}{case} \textcolor{HighlightColor}{.} \\ 
%         \textbf{Scores:} R-1: 39.26, R-2: 13.12, R-L: 22.08, MV: 56.40\\ \\
%         \textbf{\textsc{BiMeanVAE} COOP:} \ccolorbox[BackgroundColor]{ \textcolor{HighlightColor}{\strut This is}} \textcolor{HighlightColor}{a} great \textcolor{HighlightColor}{product} \ccolorbox[BackgroundColor]{ \textcolor{HighlightColor}{\strut . It}} was \ccolorbox[BackgroundColor]{ \textcolor{HighlightColor}{\strut easy to}} \ccolorbox[BackgroundColor]{ \textcolor{HighlightColor}{\strut set up}} \textcolor{HighlightColor}{and} \ccolorbox[BackgroundColor]{ \textcolor{HighlightColor}{\strut use .}} \ccolorbox[BackgroundColor]{ \textcolor{HighlightColor}{\strut The only}} downside \textcolor{HighlightColor}{is} \textcolor{HighlightColor}{that} \textcolor{HighlightColor}{it} 's \textcolor{HighlightColor}{a} little hard \textcolor{HighlightColor}{to} get \textcolor{HighlightColor}{on} \textcolor{HighlightColor}{and} off \ccolorbox[BackgroundColor]{ \textcolor{HighlightColor}{\strut , but}} \textcolor{HighlightColor}{it} 's \textcolor{HighlightColor}{not} \textcolor{HighlightColor}{a} problem \ccolorbox[BackgroundColor]{ \textcolor{HighlightColor}{\strut . It}} \underline{\ccolorbox[BackgroundColor]{ \textcolor{HighlightColor}{\strut is easy}} \ccolorbox[BackgroundColor]{ \textcolor{HighlightColor}{\strut to use}}} \textcolor{HighlightColor}{and} \textcolor{HighlightColor}{the} price \textcolor{HighlightColor}{is} right \textcolor{HighlightColor}{.}         \\ 
%         \textbf{Scores:} R-1: 28.99, R-2: 7.83, R-L: 20.11, MV: 54.33 \\ \\

%         \textbf{Optimus COOP+Attribute-Model:} \underline{\ccolorbox[BackgroundColor]{ \textcolor{HighlightColor}{\strut This is}} \ccolorbox[BackgroundColor]{ \textcolor{HighlightColor}{\strut a good}}} price \ccolorbox[BackgroundColor]{ \textcolor{HighlightColor}{\strut . You}} ca \textcolor{HighlightColor}{n't} \textcolor{HighlightColor}{be} able \ccolorbox[BackgroundColor]{ \textcolor{HighlightColor}{\strut to use}} \textcolor{HighlightColor}{it} \textcolor{HighlightColor}{in} \textcolor{HighlightColor}{your} \ccolorbox[BackgroundColor]{ \textcolor{HighlightColor}{\strut heart rate}} \ccolorbox[BackgroundColor]{ \textcolor{HighlightColor}{\strut monitor ,}} \textcolor{HighlightColor}{which} \ccolorbox[BackgroundColor]{ \textcolor{HighlightColor}{\strut is very}} \textcolor{HighlightColor}{accurate} \textcolor{HighlightColor}{.} If \ccolorbox[BackgroundColor]{ \textcolor{HighlightColor}{\strut you need}} \textcolor{HighlightColor}{to} \textcolor{HighlightColor}{be} aware \textcolor{HighlightColor}{that} \textcolor{HighlightColor}{you} have \ccolorbox[BackgroundColor]{ \textcolor{HighlightColor}{\strut to read}} all \textcolor{HighlightColor}{the} reviews \textcolor{HighlightColor}{and} \textcolor{HighlightColor}{it} 's \textcolor{HighlightColor}{very} comfortable \ccolorbox[BackgroundColor]{ \textcolor{HighlightColor}{\strut . It}} \ccolorbox[BackgroundColor]{ \textcolor{HighlightColor}{\strut is easy}} \textcolor{HighlightColor}{to} \ccolorbox[BackgroundColor]{ \textcolor{HighlightColor}{\strut set up}} \textcolor{HighlightColor}{and} \textcolor{HighlightColor}{you} \textcolor{HighlightColor}{can} \textcolor{HighlightColor}{use} \textcolor{HighlightColor}{the} \ccolorbox[BackgroundColor]{ \textcolor{HighlightColor}{\strut monitor .}} \\ 
%         \textbf{Scores:} R-1: 42.73, R-2: 12.25, R-L: 24.65, MV: 57.76 \\ \\
%         \textbf{Optimus COOP:} \textcolor{HighlightColor}{It} 's \ccolorbox[BackgroundColor]{ \textcolor{HighlightColor}{\strut very} \underline{\textcolor{HighlightColor}{easy}}}\underline{ \ccolorbox[BackgroundColor]{ \textcolor{HighlightColor}{\strut to use}}} \textcolor{HighlightColor}{.} After reading \textcolor{HighlightColor}{the} reviews \textcolor{HighlightColor}{and} \ccolorbox[BackgroundColor]{ \textcolor{HighlightColor}{\strut it is}} \textcolor{HighlightColor}{very} \textcolor{HighlightColor}{accurate} \textcolor{HighlightColor}{.} If \textcolor{HighlightColor}{you} 're looking \textcolor{HighlightColor}{for} \textcolor{HighlightColor}{a} regular basis \textcolor{HighlightColor}{,} \textcolor{HighlightColor}{you} \textcolor{HighlightColor}{can} see \ccolorbox[BackgroundColor]{ \textcolor{HighlightColor}{\strut if you}} \textcolor{HighlightColor}{need} \textcolor{HighlightColor}{to} change \textcolor{HighlightColor}{the} \ccolorbox[BackgroundColor]{ \textcolor{HighlightColor}{\strut monitor .}} \textcolor{HighlightColor}{The} \ccolorbox[BackgroundColor]{ \textcolor{HighlightColor}{\strut monitor is}} \textcolor{HighlightColor}{that} \textcolor{HighlightColor}{it} 's \textcolor{HighlightColor}{not} too bulky \ccolorbox[BackgroundColor]{ \textcolor{HighlightColor}{\strut . Very}} happy \textcolor{HighlightColor}{with} \ccolorbox[BackgroundColor]{ \textcolor{HighlightColor}{\strut this product}} \textcolor{HighlightColor}{and} \textcolor{HighlightColor}{will} \textcolor{HighlightColor}{be} reccomend \textcolor{HighlightColor}{it} \textcolor{HighlightColor}{.}  \\ 
%         \textbf{Scores:} R-1: 38.67, R-2: 8.98, R-L: 22.09, MV: 57.13 }
    
%     }
%     \caption{Vergleich der generierten Rezensionen zwischen dem COOP und COOP+Attributmodell zu Produkt B003HT9W32 des Amazon Datensatzes}
%     \label{reviewAmz2}
% \end{Rezension}


% Rezension \ref{reviewAmz2} bewertet eine Fitness Puls Uhr des Amazon Datensatzes.
% Die durch \textsc{BiMeanVAE} generierten Rezensionen sind in ihrem Inhalt konsistent und haben ein positives Sentiment. 
% Beide Rezensionen erwähnen ähnliche Eigenschaften der Bedienungsfreundlichkeit \glqq{}easy to use\grqq{}, \glqq{}easy to setup and use\grqq{} und der Funktionalität \glqq{}it works great\grqq{}, \glqq{}great product\grqq{}.
% Ebenfalls erwähnen beide generierten Rezensionen jeweils einen validen negativen Punkt der Fitness Uhr.
% Somit gehen die Rezensionen präzise auf die Aspekte der Fitness Uhr ein. 
% Die durch das Attributmodell generierte Rezension erwähnt, dass es sich um einen \glqq{}heart rate monitor\grqq{} handelt, wodurch diese Rezension noch präziser ist.
% Dies ist auch in den Metriken zu erkennen, in denen das Attributmodell in allen Metriken eine bessere Leistung als das COOP Modell erzielt.

% Die durch Optimus erzeugten Rezensionen haben eine schlechtere Konsistenz im Inhalt. 
% Hier existieren sinnvolle Teilbereiche die auf Eigenschaften eingehen wie zum Beispiel \glqq{}is very accurate\grqq{}, allerdings wiedersprechen sich in beiden Rezensionen einige Fakten und bauen nicht konsistent aufeinander auf.
% Beide Rezensionen erwähnen, dass es sich bei dem Produkt um einen \glqq{}heart rate monitor\grqq{} handelt.
% Trotz der manuell evaluierten schlechteren Konsistenz des Inhaltes weisen beide Rezensionen bessere Metriken auf als die \textsc{BiMeanVAE} Modelle, wobei auch hier bei den Optimus Modellen das Attributmodell die besten Metriken erzielt. 
% Wie auch bei Rezension \ref{reviewAmz1} zeigt hier das Attributmodell bei Optimus lediglich eine äußerst geringe Leistungssteigerung in der manuellen Evaluation, da zwar durch das Attributmodell spezielle Wörter wie \glqq{}heart rate monitor\grqq{} im Gegensatz zu \glqq{}monitor\grqq{} forciert werden, die Rezensionen trotzdem aber noch inkonsistent im Inhalt sind.

Insgesamt zeigt auf dem Amazon Datensatz das Attributmodell mit \textsc{BiMeanVAE} auf beiden Rezensionen \ref{reviewAmz1} und \ref{reviewAmz2} die besten Ergebnisse in Bezug auf konsistenten Inhalt, Semantik und präzisen Formulierungen der wichtigen Aspekte der Produkte.
Durch das Attributmodell sind die \textsc{BiMeanVAE} Rezensionen wesentlich ausdrucksstärker, da seltene Wörter, die in Bezug auf die Rezensionen eine hohe Relevanz haben, häufiger miteinbezogen werden.

\pagebreak
\subsubsection{Yelp Rezensionen} %idx 85, 67, 60 (great)
Nachfolgend werden zwei generierte Rezensionen des Yelp Datensatzes evaluiert. 
Bei dem Yelp Datensatz ist zu beachten, dass je Dienstleisung jeweils nur eine Referenzzusammenfassung existiert.
Demnach ist eine Vergleichbarkeit an Metriken nur unzureichend möglich und muss mit Vorsicht betrachtet werden, da insbesondere bei Textsequenzen Rouge Metriken die Semantik der Textsequenzen nicht erfassen.

\begin{Rezension}[!h] %60
    \centering
    %\scriptsize
    \small
    \framebox{
        \parbox{\columnwidth-4\fboxsep}{
        \textbf{Restaurant:} Harbour 60 Steakhouse \\ \\
        \textbf{Referenz:} This place has some amazing, tasty steaks!  They are breathtaking!  The menu also has seafood options that are just as delicious!  The only downside is that its pretty expensive, however you get top quality meat for what you pay for.  You won't even notice the price when your stomach is filled with the best seasoned meat you've ever had.
        \\ \\
        \textbf{\textsc{BiMeanVAE} COOP+Attribute-Model:} \ccolorbox[BackgroundColor]{ \textcolor{HighlightColor}{This place}} \textcolor{HighlightColor}{is} a great \textcolor{HighlightColor}{place} to eat\textcolor{HighlightColor}{.} \textcolor{HighlightColor}{The} food \textcolor{HighlightColor}{is} very good and \textcolor{HighlightColor}{the} service \textcolor{HighlightColor}{is} excellent\textcolor{HighlightColor}{.} It's a little pricey but it's worth it \textcolor{HighlightColor}{for} \textcolor{HighlightColor}{the} \textcolor{HighlightColor}{quality} of \textcolor{HighlightColor}{the} food and \textcolor{HighlightColor}{the} service\textcolor{HighlightColor}{.} \textcolor{HighlightColor}{The} steak was cooked to perfection and \textcolor{HighlightColor}{the} lobster bisque was \textcolor{HighlightColor}{delicious}\textcolor{HighlightColor}{.} \textcolor{HighlightColor}{The} \textcolor{HighlightColor}{best} part of \textcolor{HighlightColor}{the} meal was \textcolor{HighlightColor}{the} wine\textcolor{HighlightColor}{.}         \\ 
        \textbf{Scores:} R-1: 21.85, R-2: 5.13, R-L: 15.13, MV: 56.57\\ \\
        \textbf{\textsc{BiMeanVAE} COOP:} Excellent food and great service\textcolor{HighlightColor}{.} \textcolor{HighlightColor}{The} staff \textcolor{HighlightColor}{is} very friendly and \textcolor{HighlightColor}{the} food \textcolor{HighlightColor}{is} \textcolor{HighlightColor}{delicious}\textcolor{HighlightColor}{.} \textcolor{HighlightColor}{The} steak was cooked to perfection and \textcolor{HighlightColor}{the} service was a little slow but it was worth \textcolor{HighlightColor}{the} wait\textcolor{HighlightColor}{.} It's a great \textcolor{HighlightColor}{place} to go \textcolor{HighlightColor}{for} a date night\textcolor{HighlightColor}{.}         \\ 
        \textbf{Scores:} R-1: 18.86, R-2: 1.92, R-L: 9.43, MV: 55.69 \\ \\

        \textbf{Optimus COOP+Attribute-Model:} One of \ccolorbox[BackgroundColor]{ \textcolor{HighlightColor}{the best}} steakhouse\textcolor{HighlightColor}{.} It's a great meal\textcolor{HighlightColor}{.} \textcolor{HighlightColor}{The} service was \textcolor{HighlightColor}{amazing} and \textcolor{HighlightColor}{the} food was \textcolor{HighlightColor}{delicious}\textcolor{HighlightColor}{.} If \textcolor{HighlightColor}{you}'re looking \textcolor{HighlightColor}{for} a long time to eat at \textcolor{HighlightColor}{the} strip\ccolorbox[BackgroundColor]{\textcolor{HighlightColor}{. You}} ca\textcolor{HighlightColor}{n't} wait \textcolor{HighlightColor}{for} dinner\textcolor{HighlightColor}{.} \textcolor{HighlightColor}{This} \textcolor{HighlightColor}{is} a little pricey \textcolor{HighlightColor}{for} \textcolor{HighlightColor}{the} steak and it was perfect\textcolor{HighlightColor}{.} Highly recommend this \textcolor{HighlightColor}{place} to anyone in town\textcolor{HighlightColor}{.}         \\ 
        \textbf{Scores:} R-1: 26.89, R-2: 3.42, R-L: 15.13, MV: 57.28 \\ \\
        \textbf{Optimus COOP:} \textcolor{HighlightColor}{This} restaurant \textcolor{HighlightColor}{is} a great \textcolor{HighlightColor}{place} to die \ccolorbox[BackgroundColor]{ \textcolor{HighlightColor}{for.}} \textcolor{HighlightColor}{The} food \textcolor{HighlightColor}{is} \textcolor{HighlightColor}{amazing} and \textcolor{HighlightColor}{the} service \textcolor{HighlightColor}{is} fantastic\textcolor{HighlightColor}{.} It's a little pricey but \textcolor{HighlightColor}{the} food was \textcolor{HighlightColor}{delicious}\textcolor{HighlightColor}{.} Everything was perfect \textcolor{HighlightColor}{for} \textcolor{HighlightColor}{the} steak and \ccolorbox[BackgroundColor]{ \textcolor{HighlightColor}{delicious!}} \textcolor{HighlightColor}{The} service was great\textcolor{HighlightColor}{,} \ccolorbox[BackgroundColor]{ \textcolor{HighlightColor}{the best}} part of \textcolor{HighlightColor}{the} restaurant in \textcolor{HighlightColor}{the} city\textcolor{HighlightColor}{.} If \textcolor{HighlightColor}{you} want to go back \textcolor{HighlightColor}{for} dinner\textcolor{HighlightColor}{,} it's not worth it\textcolor{HighlightColor}{.}         \\ 
        \textbf{Scores:} R-1: 24.39, R-2: 3.30, R-L: 16.26, MV: 57.52  

        }
    }
    \caption{Vergleich der generierten Rezensionen zwischen dem COOP und COOP+Attributmodell zu Dienstleistung 4POPYEONJpkfhWOMx\_PyGg des Yelp Datensatzes} %4POPYEONJpkfhWOMx_PyGg
    \label{reviewYelp1}
\end{Rezension}

Rezension \ref{reviewYelp1} bewertet ein Steakhouse in Toronto. 
Insgesamt generieren alle Modelle gute Rezensionen und geben den Inhalt der Referenzbewertung wieder. 
Bei den durch \textsc{BiMeanVAE} generierten Bewertungen ist der Inhalt konsistent, das Essen und der Service wird als exzellent bewertet.
Die durch das Attributmodell erzeugte Rezension ist detaillierter und erwähnt zusätzlich noch den Wein und Hummer, der auch in der Referenz vorhanden ist. Ebenfalls erwähnt das Attributmodell die hohen Kosten, die auch in der Referenz vorkommen.
Die höhere Deckungsgleichheit ist auch in den Metriken wiederzufinden, in denen das Attributmodell in allen Metriken dem COOP Modell überlegen ist. 

Die Rezensionen der Optimus Modelle sind Inhaltlich korrekt und erwähnen beide, dass es sich um ein leckers Steakhouse mit gutem Service handelt.
Das COOP Modell erzeugt in dieser Rezension teilweise untypische Formulierungen wie: \glqq{}This restaurant is a great place to die for\grqq{}.
In den Metriken sind die erzeugten Rezensionen der beiden Modelle ähnlich.

\begin{Rezension}[!h] %85
    \centering
    %\scriptsize
    \small
    \framebox{
        \parbox{\columnwidth-4\fboxsep}{
        \textbf{Restaurant:} Anna Marie's Italian Cuisine \\ \\
        \textbf{Referenz:} Great place to go food authentic Italian food. Pizza is amazing, lots of topping choices  and they aren't stingy with them. Drinks and desert wont disappoint either. The atmosphere is pleasant, and provides a nice escape from the heat. Staff can be hit or miss but the place is clean and they have some good happy hour deals.
        \\ \\
        \textbf{\textsc{BiMeanVAE} COOP+Attribute-Model:} Had \textcolor{HighlightColor}{a} great experience here\ccolorbox[BackgroundColor]{\textcolor{HighlightColor}{. The}} \textcolor{HighlightColor}{food} was very \textcolor{HighlightColor}{good} \textcolor{HighlightColor}{and} \textcolor{HighlightColor}{the} service was excellent\textcolor{HighlightColor}{.} It's \ccolorbox[BackgroundColor]{ \textcolor{HighlightColor}{a nice}} \ccolorbox[BackgroundColor]{ \textcolor{HighlightColor}{place to}} \textcolor{HighlightColor}{go} \textcolor{HighlightColor}{with} \textcolor{HighlightColor}{a} group \textcolor{HighlightColor}{of} friends \textcolor{HighlightColor}{and} family\textcolor{HighlightColor}{.} They \textcolor{HighlightColor}{have} \textcolor{HighlightColor}{a} great \ccolorbox[BackgroundColor]{ \textcolor{HighlightColor}{happy hour}} menu \textcolor{HighlightColor}{and} \textcolor{HighlightColor}{the} \textcolor{HighlightColor}{food} \textcolor{HighlightColor}{is} very \textcolor{HighlightColor}{good}\textcolor{HighlightColor}{.} If you \textcolor{HighlightColor}{are} in \textcolor{HighlightColor}{the} area\textcolor{HighlightColor}{,} this \textcolor{HighlightColor}{is} \textcolor{HighlightColor}{a} must try\textcolor{HighlightColor}{.}         \\ 
        \textbf{Scores:} R-1: 41.07, R-2: 9.09, R-L: 21.43, MV: 57.87\\ \\
        \textbf{\textsc{BiMeanVAE} COOP:} Came here for \textcolor{HighlightColor}{the} first time \textcolor{HighlightColor}{and} it was \textcolor{HighlightColor}{a} great experience\ccolorbox[BackgroundColor]{\textcolor{HighlightColor}{. The}} \textcolor{HighlightColor}{food} was delicious \textcolor{HighlightColor}{and} \textcolor{HighlightColor}{the} staff was very friendly\ccolorbox[BackgroundColor]{\textcolor{HighlightColor}{. The}} \ccolorbox[BackgroundColor]{ \textcolor{HighlightColor}{atmosphere is}} \textcolor{HighlightColor}{nice} \textcolor{HighlightColor}{and} \textcolor{HighlightColor}{the} service was \textcolor{HighlightColor}{good}\textcolor{HighlightColor}{.} It's \textcolor{HighlightColor}{a} great \ccolorbox[BackgroundColor]{ \textcolor{HighlightColor}{place to}} \textcolor{HighlightColor}{go} if you \textcolor{HighlightColor}{are} in \textcolor{HighlightColor}{the} area\textcolor{HighlightColor}{.}         \\ 
        \textbf{Scores:} R-1: 33.00, R-2: 9.90, R-L: 17.47, MV: 56.94 \\ \\

        \textbf{Optimus COOP+Attribute-Model:} Came here for \textcolor{HighlightColor}{a} few weeks ago\textcolor{HighlightColor}{.} It was very friendly \textcolor{HighlightColor}{and} \textcolor{HighlightColor}{the} \textcolor{HighlightColor}{food} was delicious\textcolor{HighlightColor}{.} They \textcolor{HighlightColor}{have} \textcolor{HighlightColor}{a} great experience\ccolorbox[BackgroundColor]{\textcolor{HighlightColor}{. The}} \textcolor{HighlightColor}{food} \textcolor{HighlightColor}{is} very \textcolor{HighlightColor}{nice} \textcolor{HighlightColor}{and} it was \textcolor{HighlightColor}{a} bit \textcolor{HighlightColor}{of} \textcolor{HighlightColor}{the} best pizza\textcolor{HighlightColor}{.} If you want \ccolorbox[BackgroundColor]{ \textcolor{HighlightColor}{to go}} back \textcolor{HighlightColor}{to} try \textcolor{HighlightColor}{the} menu\textcolor{HighlightColor}{.} Definitely will definitely \textcolor{HighlightColor}{be} back \textcolor{HighlightColor}{and} try this \textcolor{HighlightColor}{place}\textcolor{HighlightColor}{.}         \\ 
        \textbf{Scores:} R-1: 35.40, R-2: 3.60, R-L: 15.93, MV: 56.91 \\ \\
        \textbf{Optimus COOP:} Absolutely love this \textcolor{HighlightColor}{place}! \textcolor{HighlightColor}{The} \textcolor{HighlightColor}{food} was very friendly \textcolor{HighlightColor}{and} \textcolor{HighlightColor}{the} service was excellent\ccolorbox[BackgroundColor]{\textcolor{HighlightColor}{. The}} staff \textcolor{HighlightColor}{is} very \textcolor{HighlightColor}{nice} \textcolor{HighlightColor}{and} attentive\textcolor{HighlightColor}{.} They \textcolor{HighlightColor}{have} \ccolorbox[BackgroundColor]{ \textcolor{HighlightColor}{a nice}} \textcolor{HighlightColor}{atmosphere} \textcolor{HighlightColor}{and} made it out \textcolor{HighlightColor}{of} \textcolor{HighlightColor}{the} menu\textcolor{HighlightColor}{.} It's \textcolor{HighlightColor}{a} must try \textcolor{HighlightColor}{to} come back\textcolor{HighlightColor}{.} Definitely will definitely \textcolor{HighlightColor}{be} back!         \\ 
        \textbf{Scores:} R-1: 34.28, R-2: 3.88, R-L: 19.04, MV: 56.75  
        }
    }
    \caption{Vergleich der generierten Rezensionen zwischen dem COOP und COOP+Attributmodell zu Dienstleistung 6jDD-Z8QcsKTdIDWwM8gog des Yelp Datensatzes}
    \label{reviewYelp2}
\end{Rezension}


In der Rezension \ref{reviewYelp2} wird ein italienisches Restaurant bewertet. 
Die Rezensionen von allen Modellen geben ein positives Feedback über das italienische Restaurant. 
Die Refrenzbewertung ist allerdings wesentlich spezifischer als \textsc{BiMeanVAE} oder Optimus in der Beschreibung des Restaurants.
\textsc{BiMeanVAE} erzeugt, sowohl als Attributmodell wie auch als COOP Modell eine in sich schlüssige Rezension. 
Die Metriken des Attributmodells sind minimal höher als die des COOP Modells.

Das Optimus Modell mit Attributionsmodell ist das einzige Modell, welches das Gericht \grqq{}best pizza\glqq{} erwähnt. 
Das normale Optimus Modell begeht den semantischen Fehler das Essen als freundlich zu beschrieben. 
Trotzdem erzeugen beide Modelle akzeptable Rezensionen und sind in den Metriken untereinander ausgeglichen.

% Insgesamt generieren alle Modelle gute Rezensionen und geben den Inhalt der Referenzbewertung wieder. 
% Bei den durch \textsc{BiMeanVAE} generierten Bewertungen ist der Inhalt konsistent, das Essen und der Service wird als exzellent bewertet.
% Die durch das Attributmodell erzeugte Rezension ist detaillierter und erwähnt zusätzlich noch den Wein und Lobster, der auch in der Referenz vorhanden ist. Ebenfalls erwähnt das Attributmodell die hohen Kosten, die auch in der Referenz vorkommen.
% Die höhere Deckungsgleichheit ist auch in den Metriken wiederzufinden, in denen das Attributmodell in allen Metriken dem COOP Modell überlegen ist. 

% Die Rezensionen der Optimus Modelle sind Inhaltlich korrekt und erwähnen beide, dass es sich um ein leckers Steakhouse mit gutem Service handelt.
% Das COOP Modell erzeugt in dieser Rezension teilweise untypische Formulierungen wie: \glqq{}This restaurant is a great place to die for\grqq{}.
% In den Metriken sind die erzeugten Rezensionen der beiden Modelle ähnlich.

Insgesamt sind die generierten Rezensionen für die beiden Restaurants des \textsc{BiMeanVAE} + Attributmodell Modells am präzisesten und im Inhalt am umfangreichsten. 
Die \textsc{BiMeanVAE} Modelle zeigen allgemein eine bessere Fähigkeit aus den Latentvektoren Text zu generieren und haben weniger semantische Fehler.
Durch das Attributmodell in Verbindung mit dem \textsc{BiMeanVAE} werden bei der Generierung noch spezifischere Details erwähnt, die durch das erneute Ranking der Tokens in Bezug auf das Attributmodell erzeugt werden.


\pagebreak

  % \section{Ergebnispräsentation}

% \begin{frame}
%   \frametitle{Grafiken}
%   Grafiken werden wie gewohnt eingebunden (nur ohne \texttt{figure}-Umgebung):
%   \begin{center}
% 	    \includegraphics[width=0.5\textwidth]{example-image-a}
%   \end{center}
% \end{frame}

% \begin{frame}{Grafiken mit Beschriftung}
% Oder alternativ mit \texttt{figure}-Umgebung:
% \begin{figure}
% \includegraphics[width=0.4\textwidth]{example-image-b}
% \caption{Ein Beispielbild}
% \end{figure}
% \end{frame}

% \begin{frame}{Quellenangaben}
% Falls es relevant ist, kann man eine Literatur-Quelle~\footfullcite{Krauthoff2017a} auf den Folien angeben.

% Dabei ist es praktisch, wenn die komplette Zitationsangaben auf der Folie selbst steht, ansonsten muss das Publikum bis zum Ende warten, um die Quellennummer aufzulösen.
% \end{frame}

% \begin{frame}{Notizen}
%   Notizen können mit \texttt{note} hinzugefügt werden. Diese werden gerendert, wenn in \texttt{master.tex} die Zeile mit \texttt{show notes on second screen} aktiviert wird.
  
%   Zum Präsentieren mit Notizen eigenen sich spezielle Programme wie PDF Presenter Console (pdfpc).
  
%   \note{Notizen}
% \end{frame}

  
  \section{Fazit}

\begin{frame}{Zusammenfassung}

  \begin{itemize}
    \item Umfassender Überblick über State-of-the-Art NLP Modelle und VAEs
    \item Erfolgreiches Optimieren von Optimus und \textsc{BiMeanVAE} zur Rezensionsgenerierung
    \begin{itemize}
      \item Bag of Words Attributmodell
      \item Rankingfunktion
    \end{itemize}
    \item $\Rightarrow$ hervorragende Performance der COOP + Attributmodell Modelle \begin{itemize} \item State-of-the-Art Ergebnisse auf dem Amazon Datensatz \item Attributmodell erwirkt eine Steigerung der Präzision \end{itemize}
  \item Ergebnisse bilden gute Grundlage für weiterführende Untersuchungen von Variational Autoencodern mit Attributmodellen
  \item Auswertung mit menschlichen Evaluatoren wäre sinnvoll, um das entwickelte Modell möglicherweise kommerziell verwenden zu können
  \end{itemize}
    
\end{frame}

% \begin{frame}{Zusammenfassung}

%   Das letzte Slide sollte eine Zusammenfassung der geleisteten Arbeit enthalten:
%   \begin{itemize}
%     \item wichtigste Resultate
%     \item größte Schwierigkeit
%     \item Future Work
%   \end{itemize}
    
%   Außerdem beachten:
%   \begin{itemize}
%     \item Auf keinen Fall überziehen (BA: 20 Minuten, MA: 30 Minuten)
%     \begin{itemize}
%         \item Präsentation mit Testpublikum üben (im Zweifel einer Quietscheente) \emph{laut}
%         \item Zeit nehmen, z.\,B. mit pdfpc oder Stoppuhr
%     \end{itemize}
%     \item Als letztes Slide Fazit stehen lassen (das ist das interessante!), kein Vielen-Dank-Slide
%     \begin{itemize}
%         \item aber trotzdem mündlich für Aufmerksamkeit danken
%     \end{itemize}
%     \end{itemize}
% \end{frame}

  \appendix

  \input{anhang}

  % % % % % % % % % % Ende der eingefügten LaTeX-Dateien % % % % % % % % % %

\end{document}

%
% Hier endet das Dokument
%
