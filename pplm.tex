\section{Kontrollierbare Textgeneration von Sprachmodellen}\raggedbottom
Die in Abschnitt \ref{coop} vorgestellte COOP Methode zur Suche der optimalen Kombination von Latentvektoren zur Maximierung des \textit{Input-Output-Overlaps} erzielt beeindruckende Resultate.
COOP basiert auf dem VAE Modell Optimus, welches zur Generation den kombinierten Latentvektor $z$ mittels Decoder $p_\theta(x|z)$ zu einem Text $\hat{x}$ rekonstruiert.

Es stellt sich die Frage ob diese Latentvektoren als Grundlage verwendet werden können, um durch weitere Optimierungen bessere Ergebnisse erzielen zu können.

Unkontrollierte Sprachmodelle modellieren Texte über die Wahrscheinlichkeit $p(X)$ für eine Sequenz $X=\{x_0,...,x_n\}$.
In Kapitel \ref{transformer} wurde die Funktionsweise von Transformer-Sprachmodellen erklärt. 
Bei der Generation werden die vorherigen Key-Value Paare der Attention-Layer in einer Vergangenheitsmatrix $H_t = [(K_t^{(1)},V_t^{(1)}), \ldots , (K_t^{(n)},V_t^{(n)})]$ gespeichert, wobei $K$ und $V$ die einzelnen Key-, Value-Vektoren im Layer $n$ zum Zeitpunkt $t$ repräsentieren. %history matrix
Diese Vergangenheitsmatrizen werden verwendet, um bei der Generation auf bereits vorher berechnete Key-, Value-Werte zurückgreifen zu können und somit effizienter Text generieren zu können.


Uber hat mit der Einführung von Plug and Play Language Models \citep{DBLP:journals/corr/abs-1912-02164} es ermöglicht, die Textgeneration bei großen Sprachmodellen wie zum Beispiel GPT-2 kontrolliert zu beeinflussen.
Kontrollierbare Generation von Texten mittels Sprachmodellen entspricht dem Modellieren von $p(x|a)$, wobei hier $a$ für ein kontrollierbares Attribut in Bezug auf den generierten Text $x$ ist. 
Mit dem Satz von Bayes lässt sich das kontrollierbare Sprachmodell zu $p(x|a)\propto p(a|x)p(x)$ umformulieren. 
Das Attribut Modell $p(a|x)$ bewertet einen Satz $x$ auf den Besitz eines Attributs $a$ mit einer Wahrscheinlichkeit.


Zur kontrollierbaren Generation werden bei PPLM-Modellen Gradienten für die generierten Sequenzen über die Log-Likelihood des normalen Sprachmodells $log(p(x))$ und der Log-Likelihood des Attribut-Modells $log(p(a|x))$ in Bezug auf die Vergangenheitsmatrix errechnet. 
Durch Veränderung der Vergangenheitsmatrix $H_t = (H_t+\Delta H_t)$ wird die Wahrscheinlichkeit das nächste Token mit den gewünschten Attributen zu erhalten erhöht. Hierbei wird $\Delta H_t$ Schrittweise durch den Gradienten des Attribut-Modells errechnet und mit Null initialisiert.
Um den Gradienten des Attribut-Modells bestimmen können wird dieses zu $p(a|H_t+\Delta H_t)$ umformuliert.
\begin{align*}
\Delta H_t \leftarrow \Delta H_t + \alpha \frac{\nabla_{\Delta H_t} \text{log }p(a|H_t+\Delta H_t)}{\| \nabla_{\Delta H_t} \text{log }p(a|H_t+\Delta H_t)\|^\gamma}
\end{align*}
In der Gleichung gibt $\alpha$ die Schrittgröße und $\gamma$ die Skalierung der Normalisierung an. Die Iteration kann merhmalig ausgeführt werden.

\subsection{Verbessern der Textgeneration von Optimus}
Den Variational Autoencoder Optimus mit einem Attributions-Modell zu kombinieren ist aufgrund der Injektion des Latentvektors schwierig.
Optimus kann bereits unter Einbezug des Latentvektors Texte kontrolliert generieren $p(x|z)$.
Die Berücksichtigung eines Attribut-Models bei der Generierung des Textes entspricht $p(x|a,z) \propto p(a|x,z)p(x|z)$.%)= \frac{p(a|x,z)p(x|z)}{p(a|z)}$. %MATH WRONG
Da der Latentvektor $z$ in die Vergangenheitsmatrix $H_t$ injeziert wird, wird der Latentvektor direkt optimiert und $\Delta z$ ergibt sich durch folgende Iteration:
\begin{align*}
    \Delta z \leftarrow \Delta z + \alpha \frac{\nabla_{\Delta z} \text{log }p(a|z+\Delta z,z)}{\| \nabla_{\Delta z} \text{log }p(a|z+\Delta z,z)\|^\gamma}
\end{align*}

%KL_LOSS
\subsection{Bag of Words Attribut-Modell}
Als Attribut Modell wird ein Bag of Words Modell verwendet, welches einen Loss über die Summe der Wahrscheinlichkeiten der einzelnen vorhergesagten Wörter bildet.
Sei $\{w_0, \ldots, w_n\}$ eine Gruppe von Tokens die ein bestimmtes Thema repräsentieren und $p_{t+1}$ die Ausgabeverteilung über die Tokens des Sprachmodells.
Dann ist die Log-Likelihood des Attribut-Modells: 
\begin{align*}
    \text{log }p(a|x) = \text{log }(\sum_{i=0}^n p_{t+1}[w_i])
\end{align*}

Um den \textit{Input-Output-Overlap} zwischen den Eingabereviews und den generierten Reviews zu maximieren wird das Bag of Words Modell erzeugt indem die $k$ am häufigsten vorkommenden Wörter über alle Reviews eines Produktes gewählt werden.
Vor dem auswählen der häufigsten vorkommenden Wörter werden von dieser Menge Stopwörter entfernt.
Die besten Ergebnisse werden mit $k=150$ erzielt. %ÜBERPRÜUFEN 

Des Weiteren können die Bag of Words Tokens gewichtet werden, indem die $k$ häufigsten Wörter nach ihrer Anzahl über eine Softmax-Funktion in eine Wahrscheinlichkeitsverteilung transformiert werden.
Hierdurch erhalten besonders häufig vorkommende Wörter ein höheres Gewicht als weniger häufig vorkommende Wörter.

Die Bag of Words Menge kann auch durch anderweitige Keyword-Extraktion wie zum Beispiel durch YAKE \citep{CAMPOS2020257} generiert werden.


Weiterhin kann die Bag of Words Menge bei der Generierung optimiert werden. 
So können bereits generierte Tokens nach einem Durchlauf aus der Bag of Words Menge entfernt werden, um zum Beispiel bei der Reviewgeneration einen Faktor nicht mehrmalig zu forcieren.


\subsection{Latentvektoroptimisierung mit Beam Search}


\subsection{Moverscore}
\label{moverscore}

\pagebreak
