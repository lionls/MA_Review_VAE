\section{Fazit}

\begin{frame}{Zusammenfassung}

  \begin{itemize}
    \item Umfassender Überblick über State-of-the-Art NLP Modelle und VAEs
    \item Erfolgreiches Optimieren von Optimus und \textsc{BiMeanVAE} zur Rezensionsgenerierung
    \begin{itemize}
      \item Bag of Words Attributmodell
      \item Rankingfunktion
    \end{itemize}
    \item $\Rightarrow$ hervorragende Performance der COOP + Attributmodell Modelle \begin{itemize} \item State-of-the-Art Ergebnisse auf dem Amazon Datensatz \item Attributmodell erwirkt eine Steigerung der Präzision \end{itemize}
  \item Ergebnisse bilden gute Grundlage für weiterführende Untersuchungen von Variational Autoencodern mit Attributmodellen
  \item Auswertung mit menschlichen Evaluatoren wäre sinnvoll, um das entwickelte Modell möglicherweise kommerziell verwenden zu können
  \end{itemize}
    
\end{frame}

% \begin{frame}{Zusammenfassung}

%   Das letzte Slide sollte eine Zusammenfassung der geleisteten Arbeit enthalten:
%   \begin{itemize}
%     \item wichtigste Resultate
%     \item größte Schwierigkeit
%     \item Future Work
%   \end{itemize}
    
%   Außerdem beachten:
%   \begin{itemize}
%     \item Auf keinen Fall überziehen (BA: 20 Minuten, MA: 30 Minuten)
%     \begin{itemize}
%         \item Präsentation mit Testpublikum üben (im Zweifel einer Quietscheente) \emph{laut}
%         \item Zeit nehmen, z.\,B. mit pdfpc oder Stoppuhr
%     \end{itemize}
%     \item Als letztes Slide Fazit stehen lassen (das ist das interessante!), kein Vielen-Dank-Slide
%     \begin{itemize}
%         \item aber trotzdem mündlich für Aufmerksamkeit danken
%     \end{itemize}
%     \end{itemize}
% \end{frame}