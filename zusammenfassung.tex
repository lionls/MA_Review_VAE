%%% Die folgende Zeile nicht ändern!
\section*{\ifthenelse{\equal{\sprache}{deutsch}}{Zusammenfassung}{Abstract}}
%%% Zusammenfassung:
Im Bereich des Natural Language Processing erzielten generative Modelle wie Variational Autoencoder große Fortschritte und ermöglichen die gesteuerte Generierung von Textsequenzen.
Insbesondere im Bereich der Multi-Document Summarization bietet sich die Verwendung von Variational Autoencodern an, um gezielt Dokumente im Latentvektorraum miteinander zu verknüpfen und präzise Zusammenfassungen dieser zu generieren.
Multi-Document Summarization bezeichnet das Zusammenfassen mehrerer Dokumente zu einem repräsentativen Dokument, welches Anwendern einen schnellen und umfassenden Überblick ermöglicht.
Im Bereich des E-Commerce und von Online-Vergleichsportalen entstehen eine große Anzahl an Rezensionen zu zahlreichen Produkten und Dienstleistungen.
So ist es für Nutzer nicht möglich, sich einen schnellen Überblick über die Besonderheiten eines Produktes oder einer Dienstleistung zu verschaffen.

Das Ziel dieser Masterarbeit ist das automatisierte, unüberwachte Zusammenfassen von mehreren Rezensionen eines Produktes oder einer Dienstleistung unter Verwendung von Variational Autoencodern zu einer kongruenten repräsentativen Rezension.
In dieser Arbeit wird ein neuer Ansatz entwickelt, der ein aktuelles Verfahren erweitert und verbessert.
Hierzu werden zwei unterschiedliche Variational Autoencoder Modelle mit einem Attributmodell optimiert.
Durch das Attributmodell werden bei der Generierung die Token Wahrscheinlichkeiten restrukturiert, um bessere Ergebnisse zu erzielen.
Zur Auswahl der generierten Rezensionen wird eine neuartige Rankingfunktion eingeführt.

Die konstruierten Modelle werden auf zwei Datensätzen ausgewertet.
In der Evaluation zeigen die entwickelten Modelle eine hervorragende Performance bei der Zusammenfassung und erreichen auf einem Datensatz neue State-of-the-Art Ergebnisse.
Somit lassen sich erfolgreich Rezensionen zu unterschiedlichen Themengebieten automatisiert mittels Variational Autoencoder mit Attributmodell zusammenfassen.


% ----


% Es werden zwei unterschiedliche Variational Autoencoder Modelle verwendet, welche mittels eines Attributmodells optimiert werden.
% Durch das Attributmodell werden bei der Generierung die Token Wahrscheinlichkeiten restrukturiert, um weitaus bessere Ergebnisse zu erzielen.

% In der Evaluation zeigen die entwickelten Modelle eine hervorragende Performance bei der Zusammenfassung und erreichen State-of-the-Art Ergebnisse auf unterschiedlichen Metriken.
% Es lassen sich Rezensionen zu unterschiedlichen Themengebieten automatisiert mittels Variational Autoencoder und Attributmodell zusammenfassen.
